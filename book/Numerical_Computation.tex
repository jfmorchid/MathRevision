\chapter{数学分析}
\thispagestyle{empty}

\setlength{\fboxrule}{0pt}\setlength{\fboxsep}{0cm}
\noindent\shadowbox{
\begin{tcolorbox}[arc=0mm,colback=lightblue,colframe=darkblue,title=Mathematical Analysis]
\kai{~~~~数值计算么么哒}

\end{tcolorbox}}
\setlength{\fboxrule}{1pt}\setlength{\fboxsep}{4pt}


\newpage

\section{背景知识}

\subsection{误差分析}

1. 误差来源:舍入误差;截断误差;模型误差;系统误差。

2. 绝对误差:$x$为准确值,$x^*$是近似值,则绝对误差$e(x^*)=x-x^*$。

3. 绝对误差限:$|e(x^*)|$的一个预估上界$\varepsilon(x^*)$称为绝对误差限。

4. 相对误差:绝对误差与准确值之比
\begin{equation*}
    e_r(x^*)=\frac{x-x^*}{x}
\end{equation*}

5. 相对误差限:$|e_r(x^*)|$的一个预估上界$\varepsilon_r(x^*)$称为相对误差限。

6. 有效数字:$x^*$左边第一个非零数字到末位数字间的位数。

\subsection{计算方法设计}

1. 误差传播:微小的误差在迭代过程中被不断放大的过程。

~~~~(1)加减法:$\varepsilon(x^*\pm y^*)=\varepsilon(x^*)+\varepsilon(y^*)$;

~~~~(2)乘法:$\varepsilon(x^*y^*)=|x^*|\varepsilon(y^*)+|y^*|\varepsilon(x^*)$;

~~~~(3)函数:$e\left(f(x^*)\right)\simeq f'(x)e(x^*)$,其中$f'(x)$称为放大因子。

2. 稳定性算法:计算过程中误差限逐步减小的算法;“病态”的不稳定算法:误差限急速增大的算法。

3. 算法设计基本原则:

~~~~(1)避免相近的数相减:会丢失有效数字;

~~~~(2)避免量级差距大的数相加减:大数吃小数;

~~~~(3)避免用绝对值小的数作除数:易产生浮点溢出;

~~~~(4)控制误差传播,确保算法稳定性。