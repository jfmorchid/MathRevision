\chapter{数学分析}
\thispagestyle{empty}

\setlength{\fboxrule}{0pt}\setlength{\fboxsep}{0cm}
\noindent\shadowbox{
\begin{tcolorbox}[arc=0mm,colback=lightblue,colframe=darkblue,title=Mathematical Analysis]
\kai{~~~~数学分析是大一新生所修的重要学科基础课, 相比非数学专业更强调证明, 对收敛性的讨论篇幅较大, 与大二的实变函数课程联系紧密. 数分
是今后多门专业课的先修课程: 积分学应用于概率论对随机变量的研究; 对积分的进一步研究(Lebesgue积分)是实变函数的重要内容; Fourier变换和多元函数积分学是偏微分方程必不可少的工具... 
山大主选教材为陈纪修的《数学分析》(第三版), 在此基础上结合卓里奇的数学分析教程, 对共计三个学期的数分课程进行完整的内容回顾.}\\

\kai{~~~~数分1的重点: 极限与连续概念, 一元函数微分学, 微分中值定理}\\

\kai{~~~~数分2的重点: 一元函数积分学, 数项级数和函数项级数, 广义积分}\\

\kai{~~~~数分3的重点: 多元函数微分学, 含参变量积分, 多元函数积分学(重积分, 曲线与曲面积分)}

\end{tcolorbox}}
\setlength{\fboxrule}{1pt}\setlength{\fboxsep}{4pt}


\newpage

\section{预备知识}

\begin{tcolorbox}[colback=red!5,colframe=red!75!black]
    ~~~~这部分主要涉及集合与函数概念,内容不多且不难, 可以认为是从高中数学到数分的过渡, 更可以认为是数学各个分支的基石.
    
    ~~~~集合论是高等数学的核心, 由此衍生出基(tu)础(tou)数学和计算机科学的区别: 一个研究连续集合, 比如实数域, 复数域等具有连续势集合上的映射, 另一个更偏向
    离散集合, 也就是有穷集和可列集上的映射. 从前者开始诞生实分析, 复分析, 傅里叶分析, 泛函分析等各大分析, 后者则衍生出图论, 组合数学, 数据结构等计算机科学分支. 认清这一点后, 
    我们便可以用一句话概括数学分析干了啥: 研究实数域或$n$维欧氏空间到实数域上的映射. 同时, 集合论又是各大学科的基础(笔者在数分, 实变, 离散数学三门课上过三遍集合论...), 故不可轻敌.

    ~~~~映射就是数学分析的研究主体. 注意到我们只研究欧氏空间到实数域的映射, 也就是实变量函数, 我们可以归纳出这一类函数的表示方法和基本性质, 同时温习一下高中数学内容.
    
\end{tcolorbox}

\subsection{集合}

1. 定义:具有某种特定性质的对象总体。

2. 集合与元素间的关系:若元素$a$在集合$A$内,则称$a$属于$A$,记作$a \in A$.

3. 集合之间的关系:若集合$A$的所有元素同时为集合$B$内的元素,则称$A$包含于$B$,记作$A \subset B$;若$A \subset B$,且$B \subset A$,则称集合$A$与集合$B$相等,记作$A=B$。

4. 集合的运算:

~~~~(1)交:$a \in A \cap B$当且仅当$a \in A$且$a \in B$,称$A \cap B$为集合$A$和集合$B$的交;

~~~~(2)并:$a \in A \cup B$当且仅当$a \in A$或$a \in B$,称$A \cup B$为集合$A$和集合$B$的并;

~~~~(3)补:$a \in A^C$当且仅当$a \notin A$,称$A^C$为集合$A$的补;

~~~~(4)差:$a \in A-B$当且仅当$a \in A$且$a \notin B$,称$A-B$为集合$A$和集合$B$的差。

5. 集合运算的性质:

~~~~(1)结合律:$A \cap(B\cap C)=(A \cap B)\cap C$;$A \cup(B\cup C)=(A \cup B)\cup C$;

~~~~(2)交换律:$A\cap B = B \cap A$;$A \cup B = B \cup A$;

~~~~(3)分配律:$A \cap (B \cup C)=(A \cap B)\cup(A\cap C)$;
$A \cup (B \cap C)=(A \cup B)\cap(A\cup C)$;

~~~~(4)德摩根律:$(A\cap B)^C=A^c\cup B^C$;$(A\cup B)^C=A^c\cap B^C$。

6. 集合的势:当集合有有限个元素时,集合的元素个数称为集合的势;当集合有无限个元素,但可以按照某种规律排成一列时,称集合具有可列势。

7. 笛卡尔积:集合$A\times B =\{(x,y)|x \in A,y \in B\}$称为集合$A$和集合$B$的笛卡尔积。

\subsection{映射}

1. 定义:若存在集合$X$与集合$Y$间的对应关系$f$,使得任意$x \in X$,存在唯一的$y \in Y$使得$f(x)=y$,
则称对应关系$f$为集合$X$到集合$Y$上的映射,记作$f:X\rightarrow Y$。此时称$x$为$y$的原像,$y$为$x$的像。

2. 特殊的映射:

~~~~(1)单射:若$f(x)=f(y)$当且仅当$x=y$,即一个像唯一对应一个原像,则称$f$为单射;

~~~~(2)满射:若对任意$y\in Y$,均存在$x \in X$,使得$f(x)=y$,则称$f$为满射;

~~~~(3)双射:若$f$既为单射又为满射,则称$f$为双射,又称为一一对应。

3. 函数:若映射$f$的定义域和值域均限制为数域,则称$f$为函数,此时$x$称为自变量,$y$称为因变量;特殊地,当$X \subset R$,$Y=R$时,称$f$为一元实函数。

4. 基本初等函数:

~~~~(1)常数函数:$y=c$,其中$c$为常数;

~~~~(2)幂函数:$y=x^\alpha$;

~~~~(3)指数函数:$y=a^x$,其中$a>0$且$a\neq 1$;

~~~~(4)对数函数:$y=\log_a x$,其中$a>0$且$a \neq 1$;

~~~~(5)三角函数:$y=\sin x$,$y=\cos x$,$y=\tan x$,$y=\cot x$;

~~~~(6)反三角函数:$y=\arcsin x$,$y=\arccos x$,$y=\arctan x$。

5. 参数方程:形如
\begin{equation*}
    \left\{\begin{aligned}
        x&=&x(t)\\
        y&=&y(t)
    \end{aligned}\right. , t\in[a,b]
\end{equation*}用参数$t$间接表示自变量和因变量关系的方程组称为参数方程。

6. 隐函数:形如$F(x,y)=0$的函数称为隐函数,通常解不出$y=f(x)$的显式函数表达式。

7. 函数的特殊性质:

~~~~(1)有界:若存在$ M>0$,使得任意$x \in X$,均有$|f(x)|\leqslant M$,则称$f(x)$为有界函数。

~~~~(2)单调:若对任意满足$x_1<x_2$的$x_1,x_2 \in X$,均有$f(x_1)\leqslant f(x_2)$,则称$f(x)$在$X$上单调递增,等号无法成立时称为严格单调递增;
若对任意满足$x_1<x_2$的$x_1,x_2 \in X$,均有$f(x_1)\geqslant f(x_2)$,则称$f(x)$在$X$上单调递减,等号无法成立时称为严格单调递减。

~~~~(3)奇偶:当$X$关于原点对称时,若对任意$x \in X$,均有$f(x)=f(-x)$,则称$f(x)$在$X$上为偶函数;
若对任意$x \in X$,均有$f(x)=-f(-x)$,则称$f(x)$在$X$上为奇函数。

~~~~(4)周期:若存在某一正数$T\in R$,使得任意$x \in X$,均有$f(x+T)=f(x)$,则称$f(x)$为周期函数,$T$为周期;符合条件的最小正数$T$称为$f(x)$的最小正周期。

\section{极限}

\begin{tcolorbox}[colback=red!5,colframe=red!75!black]
    ~~~~极限是数学分析最基本的概念,研究无限逼近时的数列和函数性质。由于极限是涉及无穷的概念,因此时常和直观感受不符。学习极限概念时,最重要的就是
    掌握如何用最严谨的$\varepsilon-N$语言描述,而不是靠着想当然来回答问题。

    ~~~~为了研究实数系上的极限,必须证明实数系的稠密性,即无限逼近的过程中,任意时刻的位置仍在实数轴上。在此基础上,根据“姚多近优多近”的思维方式取定和$\varepsilon$相关的$N$,
    严谨地推导极限的相关性质。

    ~~~~在引入极限的严谨定义后,围绕数列和函数的极限产生了实数系完备性定理,分别是确界存在性定理、单调有界准则、柯西收敛准则、闭区间套定理和Bolzano-Weierstrass定理。这五条定理是实分析严谨理论的基石,
    初学时不妨尝试着翻来覆去推导几遍。

    ~~~~温馨提示:从极限开始,会有很多需要自己动手证明的命题或者定理,用上标“*”表示。请尽量尝试独立推导和证明。

\end{tcolorbox}

\subsection{实数系连续性}

1. 数系的扩充:

~~~~(1)整数系$\mathbb{N}$:所有整数构成的集合,具有对加法运算和乘法运算的封闭性。

~~~~(2)有理数系$\mathbb{Q}$:所有形如$\frac{p}{q}$形式的有理数构成的集合,其中$p$,$q$均为非零整数。

~~~~(3)实数系$\mathbb{R}$:所有有理数和无理数构成的集合。

2. 实数系的连续性:$\forall x_1,x_2\in \mathbb{R}$,$\exists y \in \mathbb{R}$,$x_1<y<x_2$;

3. 最值:对实数集$A \in R$,若$\exists x \in A$, $\forall y \in A$,$x \geqslant y$,则称$x$为$A$的最大值;
若$\exists x \in A$, $\forall y \in A$,$x \leqslant y$,则称$x$为$A$的最小值,记作$x = \min A$。

4. 确界:若集合$A$的所有上界组成的集合为$U$,则称$\min U$为集合$A$的上确界,记作$\sup A$;
若集合$A$的所有下界组成的集合为$L$,则称$\max L$为集合$A$的下确界,记作$\inf A$。以上确界为例,证明中往往使用如下等价描述:

~~~~(1)上界:$\forall x \in A, x \leqslant \sup A$;

~~~~(2)最小:比它小的都不是$A$的上界,即$\forall \varepsilon>0$,$\exists x \in A$,$x>\sup A-\varepsilon$.

5. 确界存在定理$^*$:非空有上界的集合必存在唯一上确界,非空有下界的集合必存在唯一下确界。

\subsection{数列极限}


1. 定义:设$\{x_n\}$为实数列,$a$为实常数,若对任意给定$\varepsilon>0$,存在与$\varepsilon$有关的正整数$N$,对$\forall n>N$,均有
$\left|x_n-a\right|<\varepsilon$,则称数列$\{x_n\}$收敛于$a$,记作
\begin{equation*}
    \lim\limits_{n\rightarrow \infty}x_n=a
\end{equation*}
若不存在这样的常数$a$,则称$\{x_n\}$发散。

2. 无穷量:若$\forall G>0$,$\exists N \in \mathbb{N}^*$,$\forall n>N$,$\left|x_n \right|>G$,
则称数列$\{x_n\}$为无穷大量;若$\lim\limits_{n\rightarrow \infty}x_n=0$,则称$\{x_n\}$为无穷小量。

3. 极限的性质:

~~~~(1)唯一性$^*$:若$\lim\limits_{n\rightarrow \infty}x_n=a$,且$\lim\limits_{n\rightarrow \infty}x_n=b$,则$a=b$。

~~~~(2)有界性$^*$:若$\{x_n\}$收敛,则$\exists M >0$,$\forall n \in \mathbb{N}^*$,$\left|x_n\right|<M$。

~~~~(3)保序性$^*$:若$\lim\limits_{n\rightarrow \infty}x_n<\lim\limits_{n\rightarrow \infty}y_n$,则$\exists N\in \mathbb{N}^*$,$\forall n>N$,$x_n<y_n$。

~~~~(4)夹逼准则$^*$:若$\lim\limits_{n\rightarrow \infty}x_n=\lim\limits_{n\rightarrow \infty}z_n=a$,且$\exists N \in \mathbb{N}^*$,$\forall n>N$,$x_n\leqslant y_n\leqslant z_n$,则$\lim\limits_{n\rightarrow \infty}y_n=a$。非常重要!!

4. Stolz定理$^*$:若$\{y_n\}$是严格单调增加的正无穷大量,且
\begin{equation*}
    \lim\limits_{n\rightarrow \infty} \frac{x_n-x_{n-1}}{y_n-y_{n-1}}=a
\end{equation*}
其中$a$可为实常数或无穷大量,则$\lim\limits_{n\rightarrow \infty}\frac{x_n}{y_n}=a$。

5. 收敛判定准则:

~~~~(1)单调有界准则$^*$:若$\{x_n\}$单调递增且有上界,则$\{x_n\}$收敛;若$\{x_n\}$单调递减且有下界,则$\{x_n\}$收敛。

~~~~(2)柯西收敛准则$^*$:数列$\{x_n\}$收敛的充要条件是:$\forall \varepsilon>0$,$\exists N\in \mathbb{N}^*$,$\forall n,m>N$,$\left|x_n-x_m\right|<\varepsilon$。

6. 实数系的完备性:由实数构成的柯西列$\{x_n\}$必存在实数极限。

7. 实数系基本定理:之前介绍过确界存在性定理、单调有界准则、柯西收敛准则。

~~~(1)闭区间套定理$^*$:若一列闭区间$\{\left[a_n,b_n\right]\}$满足$\left[a_{n+1},b_{n+1}\right]\subset \left[a_{n+1},b_{n+1}\right]$,且$\lim\limits_{n\rightarrow \infty}\left(b_n-a_n\right)=0$,则存在唯一$\xi$属于所有$\left[a_n,b_n\right]$,且$\lim\limits_{n\rightarrow \infty}a_n=\lim\limits_{n\rightarrow \infty}b_n=\xi$。

~~~~(2)波尔查诺-维尔斯特拉斯定理$^*$:有界数列必有收敛子列。

\subsection{函数极限}

1. 定义:设函数$y=f(x)$在$x_0$点的某去心邻域内有定义,$A$为实常数,若对任意给定的$\varepsilon>0$,存在与$\varepsilon$有关的$\delta>0$,
使得任意$0<\left|x-x_0\right|<\delta$,均有$|f(x)-A|<\varepsilon$,则称$A$是函数$f(x)$在点$x_0$处的极限,记作$\lim\limits_{x\rightarrow x_0}f(x)=A$。

2. 性质:唯一性、局部有界性、局部保序性、夹逼准则,基本和数列极限一致。

3. 海涅定理$^*$:$\lim\limits_{x\rightarrow x_0}f(x)=A$的充分必要条件是:对任意满足$\lim\limits_{n \rightarrow \infty}x_n=x_0$的数列$\{x_n\}$,有$\lim\limits_{n \rightarrow \infty}f\left(x_n\right)=A$。

4. 单侧极限:在函数极限定义中,若范围改为$0<x-x_0<\delta$,则称为右侧极限;若范围改为$0<x_0-x<\delta$,则称为左侧极限。

5. 柯西收敛准则:函数极限$\lim\limits_{x\rightarrow +\infty} f(x)$存在的充要条件是:$\forall \varepsilon>0$,$\exists X>0$,$\forall x_1,x_2>X$,$\left|f(x_1)-f(x_2)\right|<\varepsilon$。

6. 无穷量:若$\lim\limits_{x\rightarrow x_0}f(x)=0$,则称$x\rightarrow x_0$时$f(x)$是无穷小量;
若$\lim\limits_{x\rightarrow x_0}f(x)=\infty$,则称$x\rightarrow x_0$时$f(x)$是无穷大量。

7. 无穷量的比较:以无穷小量为例,考虑两个无穷小量$f(x)$和$g(x)$的比值
\begin{equation*}
    \lim\limits_{x\rightarrow x_0}\frac{f(x)}{g(x)}=k
\end{equation*}

~~~~(1)$k=0$:$f(x)$是$g(x)$的高阶无穷小量,记作$f(x)=o\left(g(x)\right)$;

~~~~(2)$k=\infty$:$f(x)$是$g(x)$的低阶无穷小量;

~~~~(3)$k$为非零实数:$f(x)$是$g(x)$的同阶无穷小量;

~~~~(4)$k=1$:$f(x)$与$g(x)$互为等价无穷小量。

\section{连续函数}

\begin{tcolorbox}[colback=red!5,colframe=red!75!black]
    ~~~~相比于实分析、泛函分析等后继课程,数学分析的一大特点就是研究性质比较“好”的函数,比如在特定区间上连续的函数,或者可求任意阶导数的函数。诸如连续、可导、可积这种描述函数“好不好”的性质,通常称作函数分析性质。

    ~~~~在闭区间上连续的函数,性质往往比较“好”,这就意味着能从连续性出发,推导出这个函数满足的其它特性。因此,在研究连续函数的性质时,闭区间上的连续函数性质是非常重要的一环。

    ~~~~那还有没有比连续函数性质更“好”的函数呢?事实上,函数的连续性可分为“点态连续”和“一致连续”,而通常闭区间上的连续函数要求的仅仅是逐点连续。一致连续的要求更加严苛,它要求所取的$\varepsilon$必须和区间上连续点的具体位置无关。
    自然,一致连续函数也有更多的优良性质。

\end{tcolorbox}

\subsection{点态连续}

1. $f(x)$在某点$x_0$处连续的定义:若$f(x)$在$x_0$的某邻域内有定义,且$\lim\limits_{x\rightarrow x_0}f(x)=f(x_0)$,
则称函数$f(x)$在点$x_0$处连续,此时$x_0$是函数$f(x)$的一个连续点。

2. 单侧连续:若$f(x)$在$x_0$点的左极限为$f(x_0)$,则$f(x)$在$x_0$处左连续;
若$f(x)$在$x_0$点的左极限为$f(x_0)$,则$f(x)$在$x_0$处右连续。

3. 开区间上的连续性:若$f(x)$在开区间$(a,b)$的任意一点处连续,则称$f(x)$在$(a,b)$上连续。

4. 闭区间上的连续性:若$f(x)$在开区间$(a,b)$上连续,且在$a$点右连续,在$b$点左连续,则称$f(x)$在$[a,b]$上连续。

5. 性质:

~~~~(1)四则运算连续性:连续函数经过四则运算后,除去分母为零的点,仍为连续函数;

~~~~(2)反函数连续性$^*$:若函数$f(x)$在闭区间上连续且严格单增,则反函数$f^{-1}(y)$同样连续且严格单增。

~~~~(3)复合函数连续性:两个连续函数的复合仍然为连续函数。

~~~~(4)初等函数连续性:所有初等函数在定义区间上均为连续函数。

6. 间断点:若$f(x)$在点$x_0$处不连续,则称$x_0$为$f(x)$的一个间断点。根据$x_0$点处的左右极限,大致分为三类:

~~~~(1)可去间断点:左极限等于右极限,但不等于$x_0$点处的函数值;

~~~~(2)跳跃间断点:左极限不等于右极限;

~~~~(3)第二类间断点:左极限和右极限至少有一个不存在。

\subsection{闭区间上的连续函数}

1. 有界性定理$^*$:若函数$f(x)$在闭区间$[a,b]$上连续,则$f(x)$在$[a,b]$上有界。

2. 最值定理$^*$:若函数$f(x)$在闭区间$[a,b]$上连续,则$f(x)$在$[a,b]$上必能取到最大值和最小值。

3. 零点存在性定理$^*$:若函数$f(x)$在闭区间$[a,b]$上连续,且$f(a)\cdot f(b)<0$,则存在$\xi \in (a,b)$,使得$f(\xi)=0$。

4. 介值定理$^*$:若$f(x)$在闭区间$[a,b]$上连续,且最大值为$M$,最小值为$m$,则对任意$C\in[m,M]$,存在$\xi \in [a,b]$,使得$f(\xi) =C$。

\subsection{一致连续}

1. 定义:设函数$f(x)$在区间$X$上有定义,若$\forall \varepsilon>0$,$\exists \delta >0$,$\forall x_1,x_2\in X$满足$\left|x_1-x_2\right|<\delta$,
均有$\left|f(x_1)-f(x_2)\right|<\varepsilon$成立,则$f(x)$在区间$X$上一致连续。

2. 证明非一致连续的技巧:若存在区间$X$上的数列$\{x_n\}$和$\{y_n\}$满足$\lim\limits_{n\rightarrow +\infty}\left(x_n-y_n\right)=0$,但$\lim\limits_{n\rightarrow +\infty}f(x_n) \neq \lim\limits_{n\rightarrow +\infty}f(y_n)$,
则可判定$f(x)$在区间$X$上非一致连续。

3. Cantor定理:若函数$f(x)$在闭区间$[a,b]$上连续,则$f(x)$在$[a,b]$上一致连续。

\section{一元函数微分学}

\begin{tcolorbox}[colback=red!5,colframe=red!75!black]
    ~~~~在做好了关于实数系、极限、连续的铺垫后,接下来将正式进入一元函数微分学的学习。

    ~~~~高中阶段简要介绍了导数的基本求法,以及一些基本初等函数的导数,在应用层面仅仅停留在利用导数求极限部分。事实上,导数的魅力不止于此:拉格朗日中值定理揭示了弦和切线的重要联系,
    是数学证明不可或缺的重要工具;洛必达法则为比较无穷小量省去诸多繁琐的推导,仅需不断地求导,便可轻松解决不定式的问题;泰勒展开作为数值逼近的手段,在计算数学中有着极为广泛的应用。    
    
    ~~~~提醒一点,虽然之前已经掌握了基本的求导方法,但求导法则可不能乱用,导数的应用必须建立在函数可导的基础上。
    因此学习微分学的知识时,首先需要学习从定义层面上判定函数是否可导。

\end{tcolorbox}

\subsection{微分和导数的概念}

1. 微分:若自变量$x$的增量$\Delta x\rightarrow 0$时,因变量$y$的增量$\Delta y =f(x+\Delta x)-f(x)$可表示为
\begin{equation*}
    \Delta y = g(x_0)\Delta x+o(\Delta x)
\end{equation*}
则称$f(x)$在$x_0$点可微;若$f(x)$在区间$X$内任一点均可微,则称$f(x)$在区间$X$上可微,记作$\mathrm{d}y=g(x)\mathrm{d}x$。

2. 导数:若$x_0$处极限
\begin{equation*}
    \lim\limits_{\Delta x \rightarrow 0}\frac{f(x_0+\Delta x)-f(x_0)}{\Delta x}
\end{equation*}
存在,则称$f(x)$在$x_0$处可导,并称该极限值为$f(x)$在$x_0$处的导数,记作$f'(x_0)$;若$f(x)$在区间$X$内任一点均可导,则称$f(x)$在区间$X$上可导,并将导函数记作$f'(x)$或$\frac{\mathrm{d}y}{\mathrm{d}x}$。

3. 可微、可导和连续的关系:可微与可导等价,可导函数必连续,连续函数不一定可导。

4. 单侧导数:在导数定义式中,若取左极限,得到的便是左导数;若取右极限,得到的便是右导数。

5. 函数可导的判定:函数在某点可导的充要条件是该点处函数的左、右导数存在且相等。

6. 常用函数及其导数:

~~~~(1)常数函数:$(C)'=0$;

~~~~(2)指数函数、对数函数:$\left(a^x\right)'=a^x \ln a$,$\left(\log_a x\right)'=\frac{1}{x\ln a}$;

~~~~(3)幂函数;$\left(x^a\right)'=ax^{a-1}$;

~~~~(4)三角函数:$(\sin x)'=\cos x$;$(\cos x)'=-\sin x$。

7. 高阶微分:$\mathrm{d}^n y =f^{(n)}(x)\mathrm{d}x^n$,其中$f^{(n)}(x)$为对$f(x)$连续求$n$次导数的结果,又记作$f^{(n)}(x)=\frac{\mathrm{d}^n y}{\mathrm{d}x^n}$。

\subsection{求导法则}

1. 四则运算:

~~~~(1)加法和减法:$\left(f\pm g\right)'=f' \pm g'$;

~~~~(2)乘法:$(f\cdot g)'=f'\cdot g+f \cdot g'$;

~~~~(3)除法:$\left(\frac{f}{g}\right)'=\frac{f'\cdot g-f\cdot g'}{g^2}$。

2. 反函数求导法则:设反函数为$x=f(y)$,对两边求微分得$\mathrm{d}x=f'(y) \mathrm{d}y$,
整理得$\frac{\mathrm{d}y}{\mathrm{d}x}=\frac{1}{f'(y)}$。

3. 复合函数求导法则:设$y=f\left(g(x)\right)$是$y=f(u)$和$u=g(x)$两个函数的复合,则$y'(x)=y'(u)\cdot u'(x)$,或记作
\begin{equation*}
    \frac{\mathrm{d}y}{\mathrm{d}x}=\frac{\mathrm{d}y}{\mathrm{d}u}\cdot\frac{\mathrm{d}u}{\mathrm{d}x}
\end{equation*}

4. 一阶微分的形式不变性:对函数$y=f(x)$而言,无论$x$是因变量还是中间变量,一阶微分的形式总是相同的:
\begin{equation*}
    \mathrm{d}y=f'(x)\mathrm{d}x
\end{equation*}

5. 隐函数求导法则:设隐函数$F(x,y)=0$,将$y$视为关于$x$的函数$y(x)$,对两边关于$x$求导,整理出$y'$的显式函数表达式。

6. 参数方程求导法则:对参数方程$x=\varphi(t)$,$y=\psi (t)$,先求$\frac{\mathrm{d}x}{\mathrm{d}t}=\varphi'(t)$,$\frac{\mathrm{d}y}{\mathrm{d}t}=\psi'(t)$,
再将两者相除,消去$\mathrm{d}t$,得到
\begin{equation*}
    \frac{\mathrm{d}y}{\mathrm{d}x}=\frac{\psi'(t)}{\varphi'(t)}
\end{equation*}

\subsection{微分中值定理}

1. 极值:设$f(x)$在$(a,b)$上有定义,$x_0\in(a,b)$,若存在$x_0$的某一邻域,使得邻域内任意一点$x$,
均有$f(x)\leqslant f(x_0)$,则称$x_0$为$f(x)$的一个极大值点;$f(x)\geqslant f(x_0)$时,称$x_0$为$f(x)$的一个极小值点。

2. 费马引理$^*$:设$x_0$是$f(x)$的一个极值点,且$f(x)$在$x_0$处可导,则$f'(x_0)=0$。

3. 罗尔中值定理$^*$:设$f(x)$在$[a,b]$上连续,在$(a,b)$上可导,$f(a)=f(b)$,则存在$\xi \in (a,b)$,使得$f'(\xi)=0$。

4. 拉格朗日中值定理$^*$:设$f(x)$在$[a,b]$上连续,在$(a,b)$上可导,则存在$\xi \in (a,b)$,使得
\begin{equation*}
    f'(\xi)=\frac{f(b)-f(a)}{b-a}
\end{equation*}

5. 柯西中值定理$^*$:设$f(x)$和$g(x)$在$[a,b]$上连续,在$(a,b)$上可导,且$g(x)\neq 0$,则存在$\xi \in (a,b)$,使得
\begin{equation*}
    \frac{f'(\xi)}{g'(\xi)}=\frac{f(b)-f(a)}{g(b)-g(a)}
\end{equation*}

6. 洛必达法则$^*$:若$\lim\limits_{x\rightarrow x_0} f(x)$和$\lim\limits_{x\rightarrow x_0} g(x)$同为$0$或$\infty$,则有
\begin{equation*}
    \lim\limits_{x\rightarrow x_0} \frac{f(x)}{g(x)}=\lim\limits_{x\rightarrow x_0} \frac{f'(x)}{g'(x)}
\end{equation*}

\subsection{泰勒展开}

1. 带Peano余项的泰勒展开$^*$:设$f(x)$在$x_0$处有$n$阶导数,则存在$x_0$的一个邻域,对该邻域内任意一点$x$,成立
\begin{equation*}
    f(x)=f(x_0)+f'(x_0)(x-x_0)+\cdots+\frac{f^{(n)}(x_0)}{n!}(x-x_0)^n+o\left((x-x_0)^n\right)
\end{equation*}

2. 带Lagrange余项的泰勒展开:条件同上,$\xi$为$x$和$x_0$之间一点,成立
\begin{equation*}
    f(x)=f(x_0)+f'(x_0)(x-x_0)+\cdots+\frac{f^{(n)}(x_0)}{n!}(x-x_0)^n+\frac{f^{(n+1)}(\xi)}{(n+1)!}(x-x_0)^{n+1}
\end{equation*}

3. 麦克劳林公式:取$x_0=0$,其中$r_n(x)$为皮亚诺余项或拉格朗日余项:
\begin{equation*}
    f(x)=f(0)+f'(0)x+\frac{f''(0)}{2!}x^2+\cdots+\frac{f^{(n)}(0)}{n!}x^n+r_n(x)
\end{equation*}

4. 常用函数的麦克劳林展开:

~~~~(1)$e^x=1+x+\frac{x^2}{2!}+\cdots+\frac{x^n}{n!}+r_n(x)$;

~~~~(2)$\ln (1+x)=x-\frac{x^2}{2!}+\frac{x^3}{3!}\cdots+\frac{(-1)^{n+1}x^n}{n!}+r_n(x)$;

~~~~(3)$\sin x =x-\frac{x^3}{3!}+\frac{x^5}{5!}+\cdots+(-1)^n\frac{x^{2n+1}}{(2n+1)!}+r_{2n+2}(x)$;

~~~~(4)$\cos x =1-\frac{x^2}{2!}+\frac{x^4}{4!}+\cdots+(-1)^n\frac{x^{2n}}{(2n)!}+r_{2n+1}(x)$。

5. 应用举例:求未定式极限;证明和导数相关的不等式;求曲线的渐近线方程。

\section{一元函数积分学}

\begin{tcolorbox}[colback=red!5,colframe=red!75!black]
    ~~~~不定积分是和微分相对应的概念。就和乘法、除法的关系一样,不定积分就是把微分作逆运算,倒回去求原函数。定积分则是
    用于求不规则体面积的工具,将横轴分为若干小段,在每一段内求矩形面积再求和。

    ~~~~这两个出发点完全不同的概念,仅仅用了一个公式就紧密地联系在一起,这就是一元函数微积分中最伟大的结论之一:Newton-Leibniz公式。
    有了这个公式后,定积分和不定积分的求解方法便可完全互通,知道不定积分和上下限,便能轻松求出不规则区域的面积。

    ~~~~这个公式的意义不止于此,若将积分上下限改为间断点或无穷远处,仍可以通过微积分基本定理转化为不定积分求解问题。这种积分限较为特殊的积分称为广义积分,除了关心
    广义积分的具体值外,往往还会研究广义积分的收敛性,并引申出很多收敛性的判别法。
\end{tcolorbox}

\subsection{不定积分}

1. 原函数:若函数$F(x)$和$f(x)$满足$F'(x)=f(x)$,则称$F(x)$为$f(x)$的原函数。

2. 不定积分:函数$f(x)$原函数的全体称为$f(x)$的不定积分,记作$\int f(x)\mathrm{d}x$。
若$f(x)$的某一个原函数为$F(x)$,则$\int f(x)\mathrm{d}x = F(x)+C$,其中$C$为任意常数。

3. 第一类换元积分法:若$f(x)$可等价变化为$f\left(u(x)\right)\cdot u'(x)$,则用$\mathrm{d}u=u'(x)\mathrm{d}x$代换,得
\begin{equation*}
    \int f(x)\mathrm{d} x =\int f(u)\mathrm{d}u
\end{equation*}
即用中间变量$u$代替原本的变量$x$进行积分。当前的积分变元由微分是$\mathrm{d}x$还是$\mathrm{d}u$决定,下同。

4. 第二类换元积分法:代入$\mathrm{d}x =x'(u)\mathrm{d}u$,得
\begin{equation*}
    \int f(x)\mathrm{d} x =\int f\left(x(u)\right)\cdot x'(u)\mathrm{d} u 
\end{equation*}
即把原本变量$x$替换为中间变量$u$积分。

5. 分部积分法:利用微分的乘法法则$\mathrm{d}(u\cdot v)=u\cdot \mathrm{d}v +v\cdot \mathrm{d}u$,两边求积分得
\begin{equation*}
    u\cdot v =\int u \mathrm{d}v+\int v\mathrm{d}u
\end{equation*}
因此,在求$\int f(x)\mathrm{d} g(x)$时,可以借助分部积分,转化为$f(x)\cdot g(x)-\int g(x)\mathrm{d}f(x)$,交换待积式和微分式。

\subsection{定积分}

1. 定义:设$f(x)$是定义在$[a,b]$上的有界函数,在$[a,b]$上任意取划分$P:a=x_0<x_1\cdots<x_n=b$,
在区间$[x_{i-1},x_i]$内任取一点$\xi_i$。记小区间$[x_{i-1},x_i]$的长度为$\Delta x_i=x_i-x_{i-1}$,并记小区间长度的最大值为$\lambda=\max\limits_{1\leqslant i\leqslant n} \Delta x_i$。
若如下极限存在:
\begin{equation*}
    \lim\limits_{\lambda \rightarrow 0} \sum\limits_{i=1}^n f\left(\xi_i\right)\Delta x_i
\end{equation*}
且极限值与划分$P$和点$\xi$的取法均无关,则称该极限值为$f(x)$在$[a,b]$上的定积分,记作$\int_a^b f(x)\mathrm{d}x$,此时称$f(x)$在$[a,b]$上可积。

2. 可积性条件:

~~~~(1)达布和:令$M_i=\max\limits_{x_{i-1}\leqslant x\leqslant x_i}f(x)$,$m_i=\min\limits_{x_{i-1}\leqslant x\leqslant x_i}f(x)$,
则称$\sum\limits_{i=1}^n M_i\Delta x_i$为达布大和,$\sum\limits_{i=1}^n m_i\Delta x_i$为达布小和。

~~~~(2)黎曼可积的充要条件:$f(x)$在$[a,b]$上可积的充要条件是:对任意划分$P$,当区间长度的最大值$\lambda \rightarrow 0 $时,达布大和与达布小和的极限值相等。

~~~~(3)振幅与可积性条件:记振幅$\omega_i=M_i-m_i$,则有界函数$f(x)$对任意划分$P$,当区间长度的最大值$\lambda \rightarrow 0 $时,$\lim\limits_{\lambda \rightarrow 0}\sum\limits_{i=1}^n \omega_i\Delta x_i= 0$。

~~~~(4)闭区间上函数的可积性$^*$:闭区间上的连续函数必定可积;闭区间上的单调函数必定可积。

~~~~(5)可积的常用判定方法$^*$:有界函数$f(x)$在$[a,b]$上可积的充要条件是:对任意给定的$\varepsilon>0$,存在一种划分$P$,使得$\sum\limits_{i=1}^n \omega_i \Delta x_i <\varepsilon$。

3. 定积分的性质:

~~~~(1)线性性质:$\int_a^b \left(k_1f(x)+k_2g(x)\right)\mathrm{d}x=k_1\int_a^bf(x)\mathrm{d}x+k_2\int_a^bg(x)\mathrm{d}x$;

~~~~(2)保序性$^*$:若$[a,b]$上恒有$f(x)\geqslant g(x)$,则$\int_a^b f(x)\mathrm{d}x\geqslant \int_a^b g(x)\mathrm{d}x$。

~~~~(3)绝对可积性$^*$:若$f(x)$在$[a,b]$上可积,则$\left|f(x)\right|$在$[a,b]$上也可积,且成立
\begin{equation*}
    \left|\int_a^bf(x)\mathrm{d}x\right|\leqslant \int_a^b \left|f(x)\right|\mathrm{d}x
\end{equation*}

~~~~(4)区间可加性$^*$:若$c\in [a,b]$,则$f(x)$在$[a,b]$上可积的充要条件是$f(x)$在$[a,c]$和$[c,b]$上均可积,且满足
\begin{equation*}
    \int_a^b f(x)\mathrm{d}x=\int_a^c f(x)\mathrm{d}x+\int_c^b f(x)\mathrm{d}x
\end{equation*}

4. 积分第一中值定理$^*$:

~~~~(1)若$f(x)$和$g(x)$在区间$[a,b]$上可积,$g(x)$在$[a,b]$上不变号,记$\sup\limits_{[a,b]}f(x)=M$,$\inf\limits_{[a,b]}f(x)=m$,则存在$\eta \in [m,M]$,满足
\begin{equation*}
    \int_a^b f(x)g(x)\mathrm{d}x=\eta \int_a^b g(x)\mathrm{d}x
\end{equation*}

~~~~(2)在上述条件下,若$f(x)$在$[a,b]$上连续,则存在$\xi \in [a,b]$,使得
\begin{equation*}
    \int_a^b f(x)g(x)\mathrm{d}x =f(\xi) \int_a^b g(x)\mathrm{d}x
\end{equation*}

5. 变限积分:设$f(x)$在$[a,b]$上连续,作函数$F(x)=\int_a^x f(t)\mathrm{d}t$,其中$x \in [a,b]$,
则该函数称为$f(x)$的变上限积分,同理定义$F(x)=\int_x^b f(t)\mathrm{d}t$为$f(x)$的变下限积分。
此时$F(x)$在$[a,b]$上为可微函数,且$F'(x)=f(x)$。

6. 微积分基本定理:设$f(x)$在$[a,b]$上连续,$F(x)$是$f(x)$在$[a,b]$上的一个原函数,则有
\begin{equation*}
    \int_a^b f(x)\mathrm{d}x=F(b)-F(a)
\end{equation*}
此公式便是著名的牛顿-莱布尼茨公式。

7. 定积分用于几何计算:

~~~~(1)$f(x)$与$x$轴围成的平面图形面积:设图形所处横坐标区间为$[a,b]$,则该面积为$\int_a^b f(x)\mathrm{d} x$;

~~~~(2)$f(x)$图像上一段曲线的弧长:设曲线所处横坐标区间为$[a,b]$,则曲线弧长为$\int_a^b \sqrt{1+\left[f'(x)\right]^2}\mathrm{d}x$;

~~~~(3)$f(x)$围绕$x$轴旋转一圈形成的旋转体体积:设旋转体所处横坐标区间为$[a,b]$,则旋转体体积为$\pi \int_a^b \left[f(x)\right]^2\mathrm{d}x$;

~~~~(4)极坐标方程$r=r(\theta)$下平面图形面积:设图形所处角度区间为$[\alpha,\beta]$,则面积为$\frac{1}{2}\int_\alpha^\beta r^2(\theta)\mathrm{d}\theta$。

\subsection{广义积分}

1. 定义:积分区间无限或者被积函数无界的定积分。

2. 收敛性:设函数$f(x)$在$[a,+\infty]$上有定义,且在任意有限区间$[a,A]$上可积。若极限
\begin{equation*}
    \lim\limits_{A\rightarrow +\infty}\int_a^A f(x)\mathrm{d}x
\end{equation*}
存在,则称广义积分$\int_a^{+\infty}f(x)\mathrm{d}x$收敛,积分值等于上述极限值;若该极限不存在,则称广义积分$\int_a^{+\infty}f(x)\mathrm{d}x$发散。

3. 柯西主值:若
\begin{equation*}
    \lim\limits_{A\rightarrow +\infty}\int_{-A}^A f(x)\mathrm{d}x=\lim\limits_{A\rightarrow+\infty} \left[F(A)-F(-A)\right]
\end{equation*}

收敛,则称该极限为$\int_{-\infty}^{+\infty}f(x)\mathrm{d}x$的柯西主值,记作(cpv)$\int_{-\infty}^{+\infty}f(x)\mathrm{d}x$。

4. 柯西收敛准则:广义积分$\int_a^{+\infty}f(x)\mathrm{d}x$收敛的充分必要条件是:$\forall \varepsilon>0$,$\exists N \geqslant a$,$\forall A_1,A_2>N$,有
\begin{equation*}
    \left|\int_{A_1}^{A_2}f(x)\mathrm{d}x\right|<\varepsilon
\end{equation*}

5. 绝对收敛:若广义积分$\int_a^{+\infty}\left|f(x)\right|\mathrm{d}x$收敛,则称$\int_a^{+\infty}f(x)\mathrm{d}x$绝对收敛;
若$\int_a^{+\infty}\left|f(x)\right|\mathrm{d}x$发散而$\int_a^{+\infty}f(x)\mathrm{d}x$收敛,则称$\int_a^{+\infty}f(x)\mathrm{d}x$条件收敛。

6. 非负函数广义积分的收敛判别法:

~~~~(1)比较判别法:若$[a,+\infty)$上恒有$0\leqslant f(x)\leqslant K\varphi(x)$,则$\int_a^{+\infty}\varphi(x)\mathrm{d}x$收敛时 $\int_a^{+\infty}f(x)\mathrm{d}x$收敛,$\int_a^{+\infty}f(x)\mathrm{d}x$发散时
$\int_a^{+\infty}\varphi(x)\mathrm{d}x$发散。

~~~~(2)比较判别法的极限形式:考虑$\lim\limits_{x\rightarrow+\infty}\frac{f(x)}{\varphi(x)}=l$,当$l$为实常数时$\int_a^{+\infty}f(x)\mathrm{d}x$
和$\int_a^{+\infty}\varphi(x)\mathrm{d}x$同时收敛或发散;当$l=\infty$时,$\int_a^{+\infty}\varphi(x)\mathrm{d}x$发散意味着$\int_a^{+\infty}f(x)\mathrm{d}x$发散;
当$l=0$时,$\int_a^{+\infty}\varphi(x)\mathrm{d}x$收敛意味着$\int_a^{+\infty}f(x)\mathrm{d}x$收敛。

~~~~(3)柯西判别法:若$f(x)\leqslant\frac{K}{x^p}$,$p>1$,则$\int_a^{+\infty}f(x)\mathrm{d}x$收敛;若$f(x)\geqslant \frac{K}{x^p}$,$p\leqslant 1$,则$\int_a^{+\infty}f(x)\mathrm{d}x$发散。

7. 积分第二中值定理:设$f(x)$在$[a,b]$上可积,$g(x)$在$[a,b]$上单调,则存在$\xi \in [a,b]$,使得
\begin{equation*}
    \int_a^b f(x)g(x)\mathrm{d}x=g(a)\int_a^\xi f(x)\mathrm{d}x+g(b)\int_\xi ^b f(x)\mathrm{d}x
\end{equation*}

8. Abel-Dirichlet判别法:若下面某个条件满足,则广义积分$\int_a^{+\infty} f(x)g(x)\mathrm{d}x$收敛:

~~~~(1)Abel条件:$\int_a^{+\infty}f(x)\mathrm{d}x$收敛,$g(x)$在$[a,+\infty)$上单调有界;

~~~~(2)Dirichlet条件:$F(A)=\int_a^Af(x)\mathrm{d}x$在$A\in[a,+\infty)$上有界,$g(x)$在$[a,+\infty)$上单调且$\lim\limits_{x\rightarrow +\infty}g(x)=0$。

\section{无穷级数}

\begin{tcolorbox}[colback=red!5,colframe=red!75!black]
    ~~~~无穷级数研究的目标是无穷项的和。当求和的每一项都是确定的数时,该级数被称作数项级数;当求和的项和某一自变量$x$相关时,求和的结果会是关于$x$的函数,被称作函数项级数。
    
    ~~~~数项级数分为两种情况:当求和结果随着项数增加而越来越逼近某一特定值时,该级数收敛;当求和结果无法控制在某一具体值附近时,该级数发散。因此,研究常数项级数的性质时,不仅要掌握
    常数项级数的求法,还需要掌握判断级数敛散性的方法。

    ~~~~函数项级数也分为两种情况:点态收敛和一致收敛。在$x$的定义域内,若任取一点$x_0$代入,得到的数项级数都收敛,则函数项级数是点态收敛的;一致收敛的条件更为苛刻,性质也更为优异。

    ~~~~傅里叶级数可以将任意信号转化为正弦、余弦信号的叠加,
    由此衍生出的傅里叶变换、离散余弦变换、快速傅里叶变换提供了时域-频域转化的方法,是音频和图像信号处理的一大利器。

\end{tcolorbox}

\subsection{数项级数}

1. 定义:记部分和数列$S_n=\sum\limits_{i=1}^n x_i$,若$\{S_n\}$收敛于有限数$S$,则称无穷级数$\sum\limits_{n=1}^\infty x_n$收敛,记作
\begin{equation*}
    \sum\limits_{n=1}^\infty x_n = S
\end{equation*}
若部分和数列$\{S_n\}$发散,则称无穷级数$\sum\limits_{n=1}^\infty x_n$发散。

2. 性质$^*$:若$\sum\limits_{n=1}^\infty x_n$收敛,则必有$\lim\limits_{n\rightarrow\infty} x_n=0$。

3. 正项级数:

~~~~(1)定义:每一项均为非负实数的级数,即$\forall n\in \mathbb{R}$,$x_n \geqslant 0$,则称$\sum\limits_{n=1}^\infty x_n$为正项级数。

~~~~(2)收敛原理:正项级数收敛的充要条件是它的部分和数列有上界。

~~~~(3)比较判别法$^*$:若存在$N\in \mathbb{N}^*$和常数$A>0$,使得$n>N$时恒有$x_n\leqslant A y_n$,则$\sum\limits_{n=1}^\infty y_n$收敛意味着$\sum\limits_{n=1}^\infty x_n$收敛,
$\sum\limits_{n=1}^\infty x_n$发散意味着$\sum\limits_{n=1}^\infty y_n$发散。

~~~~(4)比较判别法的极限形式:设$\lim\limits_{n\rightarrow\infty}\frac{x_n}{y_n}=l$,若$l$为实常数,则$\sum\limits_{n=1}^\infty x_n$和$\sum\limits_{n=1}^\infty y_n$同敛散性;若$l=0$,
则$\sum\limits_{n=1}^\infty y_n$收敛意味着$\sum\limits_{n=1}^\infty x_n$收敛;若$l=\infty$,则$\sum\limits_{n=1}^\infty y_n$发散意味着$\sum\limits_{n=1}^\infty x_n$发散。

~~~~(5)柯西判别法$^*$:考虑$r=\lim\limits_{n\rightarrow\infty}\sqrt[n]{x_n}$。若$r>1$,则$\sum\limits_{n=1}^\infty x_n$发散;若$r<1$,则$\sum\limits_{n=1}^\infty x_n$收敛;若$r=1$,则柯西判别法失效。

~~~~(6)达朗贝尔判别法$^*$:考虑$r=\lim\limits_{n\rightarrow\infty}\frac{x_{n+1}}{x_n}$。若$r>1$,则$\sum\limits_{n=1}^\infty x_n$发散;若$r<1$,则$\sum\limits_{n=1}^\infty x_n$收敛;若$r=1$,则达朗贝尔判别法失效。

~~~~(7)拉比判别法:考虑$r=\lim\limits_{n\rightarrow\infty}n\left(\frac{x_n}{x_{n+1}}\right)$。若$r>1$,则级数$\sum\limits_{n=1}^\infty x_n$收敛;若$r<1$,则级数$\sum\limits_{n=1}^\infty x_n$发散;若$r=1$,则拉比判别法失效。

~~~~(8)广义积分判别法:正项级数$\sum\limits_{n=1}^\infty \left[\int_{a_n}^{a_{n+1}}f(x)\mathrm{d}x\right]$与广义积分$\int_a^{+\infty}f(x)\mathrm{d}x$敛散性一致,其中$\lim\limits_{n\rightarrow\infty} a_n=+\infty$。

4. 任意项级数:

~~~~(1)柯西收敛准则:级数$\sum\limits_{n=1}^\infty x_n$收敛的充要条件是:$\forall \varepsilon>0$,$\exists N \in \mathbb{N}^*$,$\forall m>n>N$,$\left|\sum\limits_{k=n}^m x_k\right|<\varepsilon$。

~~~~(2)莱布尼茨判别法$^*$:若正项级数$\{x_n\}$单调递减且收敛于$0$,则交错级数$\sum\limits_{n=1}^\infty (-1)^n x_n$收敛。

~~~~(3) Abel-Dirichlet判别法:若下面某个条件满足,则无穷级数$\sum\limits_{n=1}^\infty a_nb_n$收敛:

~~~~~~~~~~~~$\cdot$ Abel条件:$\sum\limits_{n=1}^\infty b_n$收敛,$\{a_n\}$单调有界;

~~~~~~~~~~~~$\cdot$ Dirichlet条件:$\sum\limits_{n=1}^\infty b_n$有界,$\{a_n\}$单调趋于$0$。

~~~~(4)绝对收敛:若级数$\sum\limits_{n=1}^\infty \left|x_n\right|$收敛,则称$\sum\limits_{n=1}^\infty x_n$绝对收敛;若$\sum\limits_{n=1}^\infty \left|x_n\right|$发散,且$\sum\limits_{n=1}^\infty x_n$收敛,则称$\sum\limits_{n=1}^\infty x_n$条件收敛。

\subsection{函数项级数}

1. 定义:设$u_n(x)$是具有公共定义域的一列函数,对无穷个函数求和的结果$\sum\limits_{n=1}^{\infty} u_n(x)$称为函数项级数。

2. 点态收敛性:若对定义域内给定一点$x_0$,数项级数$\sum\limits_{n=1}^{\infty} u_n(x_0)$收敛,则称$\sum\limits_{n=1}^{\infty} u_n(x)$在点$x_0$处收敛。$\sum\limits_{n=1}^{\infty} u_n(x)$所有收敛点的全体称为$\sum\limits_{n=1}^{\infty} u_n(x)$的收敛域。

3. 和函数:若$\sum\limits_{n=1}^{\infty} u_n(x)$的收敛域为$D$,则$\sum\limits_{n=1}^{\infty} u_n(x)$在$D$上定义了一个关于$x$的函数$S(x)=\sum\limits_{n=1}^{\infty} u_n(x)$,称作$\sum\limits_{n=1}^{\infty} u_n(x)$的和函数。

4. 一致收敛性:

~~~~(1)定义:若对任意给定的$\varepsilon>0$,存在仅与$\varepsilon$有关而与$x$无关的$N\in \mathbb{N}^*$,当$n>N$时,
\begin{equation*}
    \left|S_n(x)-S(x)\right|<\varepsilon
\end{equation*}
则称$\{S_n(x)\}$一致收敛于函数$S(x)$。

~~~~(2)内闭一致收敛:若对任意闭区间$[a,b]\subset D$,均有$\{S_n(x)\}$在$[a,b]$上一致收敛于$S(x)$,则称$\{S_n(x)\}$在$D$上内闭一致收敛于$S(x)$。

~~~~(3)判定定理:设$\{S_n(x)\}$在$D$上点态收敛于$S(x)$,定义$\{S_n(x)\}$与$S(x)$的距离为$d(S_n,s)=\sup\limits_{x\in D} \left|S_n(x)-S(x)\right|$,则$\{S_n(x)\}$在$D$上一致收敛于$S(x)$的充分必要条件是:
\begin{equation*}
    \lim\limits_{n\rightarrow\infty}d(S_n,S)=0
\end{equation*}

~~~~(4)非一致收敛判定:若存在数列$\{x_n\}$,$x_n\in D$,使$\lim\limits_{n\rightarrow\infty}\left(S_n(x_n)-S(x_n)\right)\neq 0$,则$\{S_n(x)\}$在$D$上不一致收敛于$S(x)$。

5. 一致收敛级数:

~~~~(1)定义:记$S_n(x)$为$\{u_n(x)\}$的部分和数列,若$S_n(x)$一致收敛于$S(x)$,则函数项级数$\sum\limits_{n=1}^{\infty} u_n(x)$一致收敛于$S(x)$。

~~~~(2)柯西收敛准则:$\sum\limits_{n=1}^{\infty} u_n(x)$在$D$上一致收敛的充要条件是:$\forall \varepsilon>0$,$\exists N \in  \mathbb{N}^*$,$\forall m>n>N$,$x\in D$,$\left|u_{n+1}(x)+\cdots+u_m(x)\right|<\varepsilon$。

~~~~(3)维尔斯特拉斯判别法:若$\forall x\in D$,$\left|u_n(x)\right|\leqslant a_n$,且$\sum\limits_{n=1}^{\infty} a_n$收敛,则$\sum\limits_{n=1}^{\infty} u_n(x)$在$D$上一致收敛。

~~~~(4) Abel-Dirichlet判别法:若下面某条件满足,则函数项级数$\sum\limits_{n=1}^\infty a_n(x)b_n(x)$在$D$上一致收敛:

~~~~~~~~~~~~$\cdot$ Abel条件:$\sum\limits_{n=1}^\infty b_n(x)$在$D$上一致收敛,且对任意$x\in D$,$\{a_n(x)\}$关于$n$单调且一致有界;

~~~~~~~~~~~~$\cdot$ Dirichlet条件:$\sum\limits_{n=1}^\infty b_n(x)$的部分和序列在$D$上一致有界,且对任意$x \in D$,$\{a_n(x)\}$关于$n$单调且一致收敛于$0$。

~~~~(5)性质:可逐项积分;可逐项求极限;可逐项求导。

6. 幂级数:

~~~~(1)定义:形如$\sum\limits_{n=0}^\infty a_nx^n$的函数项级数称为幂级数。

~~~~(2)收敛域:定义收敛半径$R=\frac{1}{\lim\limits_{n\rightarrow\infty}\sqrt[n]{|a_n|}}$,当$|x|<R$时幂级数收敛,$|x|>R$时幂级数发散,$|x|=R$时单独代入判断。
使幂级数收敛的$x$全体称为幂级数的收敛域。

~~~~(3)达朗贝尔判别法:若$\lim\limits_{n\rightarrow\infty}\left|\frac{a_{n+1}}{a_n}\right|=A$,则幂级数的收敛半径为$R=\frac{1}{A}$。

~~~~(4)性质:幂级数在其收敛域上连续,在收敛域内部可逐项求导、逐项积分。

7. 泰勒级数:当泰勒展开
\begin{equation*}
    f(x)=f(x_0)+f'(x_0)(x-x_0)+\cdots+\frac{f^{(n)}(x_0)}{n!}(x-x_0)^n+r_n(x)
\end{equation*}

中,余项满足$\lim\limits_{n\rightarrow\infty}r_n(x)=0$时,$f(x)$可展开成幂级数$\sum\limits_{n=0}^\infty\frac{f^{(n)}(x_0)}{n!}(x-x_0)^n$,且幂级数收敛于函数$f(x)$。

\subsection{傅里叶级数}

1. 定义:设$f(x)$是以$2\pi$为周期的函数,且在$[-\pi,\pi]$上可积或绝对可积,则$f(x)$可展开为如下Fourier级数:
\begin{equation*}
    f(x)\sim \frac{a_0}{2}+\sum\limits_{n=1}^\infty \left(a_n\cos nx+b_n\sin nx\right)
\end{equation*}
其中$a_n=\frac{1}{\pi}\int_{-\pi}^\pi f(x)\cos nx\mathrm{d}x$,$b_n=\frac{1}{\pi}\int_{-\pi}^\pi f(x)\sin nx\mathrm{d}x$。

2. 正弦级数和余弦级数:若$f(x)$为奇函数,则$a_n=0$,即$f(x)$可展开为正弦级数$\sum\limits_{n=1}^\infty b_n\sin nx$;
若$f(x)$为偶函数,则$b_n=0$,即$f(x)$可展开为余弦级数$\frac{a_0}{2}+\sum\limits_{n=1}^\infty a_n\sin nx$。

3. 任意周期函数的Fourier展开:设$f(x)$是以$2T$为周期的函数,且在$[-T,T]$上可积或绝对可积,则$f(x)$可展开为如下Fourier级数:
\begin{equation*}
    f(x)\sim \frac{a_0}{2}+\sum\limits_{n=1}^\infty \left(a_n\cos \frac{n\pi}{T}x+b_n\sin \frac{n\pi}{T}x\right)
\end{equation*}
其中$a_n=\frac{1}{T}\int_{-T}^T f(x)\cos \frac{n\pi}{T}x\mathrm{d}x$,$b_n=\frac{1}{T}\int_{-T}^T f(x)\sin \frac{n\pi}{T}x\mathrm{d}x$。

4. 单点处的收敛性:若$f(x)$在$[-\pi,\pi]$上可积或绝对可积,在点$x$处存在两个单侧导数,则$f(x)$的傅里叶级数在点$x$处收敛于$\frac{f(x+)+f(x-)}{2}$。

5. 性质:可逐项积分;可逐项微分。

6. 傅里叶变换:
\begin{equation*}
    \mathscr{F}(f)=\int_{-\infty}^{+\infty}f(x)e^{-i\omega x}\mathrm{d}x
\end{equation*}
称为$f(x)$的傅里叶变换;而
\begin{equation*}
    \mathscr{F}^{-1}(f)=\frac{1}{2\pi}\int_{-\infty}^{+\infty}f(\omega)e^{i\omega x}\mathrm{d}\omega
\end{equation*}
称为$f(\omega)$的傅里叶逆变换。

7. 卷积定理:记$f(x)$和$g(x)$的卷积为
\begin{equation*}
    (f*g)(x)=\int_{-\infty}^{+\infty} f(t)g(x-t)\mathrm{d}t
\end{equation*}
则傅里叶变换将卷积化为乘积,即$\mathscr{F}[f*g]=\mathscr{F}[f]\cdot \mathscr{F}[g]$。

\section{多元函数微分学}

\begin{tcolorbox}[colback=red!5,colframe=red!75!black]
    ~~~~多元函数是指欧几里得空间$\mathbb{R}^n$的子集到$R$上的映射。由于高维欧氏空间的部分概念如极限、连续、区间都无法直接套用实数域上的定义,
    因此在引入多元函数的偏导数和全微分之前,有必要重新定义欧氏空间上的部分基础概念。

    ~~~~偏导数就是多元函数针对某一坐标轴的方向导数,如$\frac{\partial f}{\partial x}$就固定$y$坐标,只对$x$方向求导。在此基础上,定义了多元函数的全微分,并指出多元函数的可微和可偏导不是对等的。
    偏导数还有多种求导法则,如针对复合函数的链式法则,以及针对隐函数的逆映射定理。

    ~~~~偏导数的应用非常广泛。在空间解析几何中,利用偏导数可以求曲线的切线和法平面,以及曲面的切平面与法线;在最优化问题中,不仅可以求无条件限制的极值,还可用拉格朗日乘子法求解条件极值。
\end{tcolorbox}

\subsection{欧几里得空间}

1. 定义:设$\mathbb{R}^n={(x_1,\cdots,x_n)|x_i\in R}$为$n$个$\mathbb{R}$的笛卡尔积,在此基础上定义加法运算、数乘运算、向量内积和距离,即构成欧氏空间,其中的元素称为向量。

~~~~(1)加法运算:设$\mathbf{x}=(x_1,\cdots,x_n)$,$\mathbf{y}=(y_1,\cdots,y_n)$,定义$\mathbf{x}+\mathbf{y}=(x_1+y_1,\cdots,x_n+y_n)$;

~~~~(2)数乘运算:设$\lambda\in \mathbb{R}$,$\mathbf{x}=(x_1,\cdots,x_n)$,定义$\lambda\mathbf{x}=(\lambda x_1,\cdots,\lambda x_n)$;

~~~~(3)内积:定义$\mathbf{x}$和$\mathbf{y}$的内积$\left\langle\mathbf{x},\mathbf{y}\right\rangle=x_1y_1+\cdots+x_ny_n=\sum\limits_{i=1}^n x_iy_i$;

~~~~(4)距离:定义$\mathbf{x}$和$\mathbf{y}$的距离$|\mathbf{x}-\mathbf{y}|=\sqrt{(x_1-y_1)^2+\cdots+(x_n-y_n)^2}$;

~~~~(5)范数:定义$\|\mathbf{x}\|=\sqrt{\left\langle\mathbf{x},\mathbf{x}\right\rangle}=\sqrt{x_1^2+\cdots+x_n^2}$为$\mathbf{x}$的Euclid范数。

2. 内积的性质:

~~~~(1)正定性:$\left\langle\mathbf{x},\mathbf{x}\right\rangle\geqslant 0$,当且仅当$\mathbf{x}=\mathbf{0}$时取等;

~~~~(2)对称性:$\left\langle\mathbf{x},\mathbf{y}\right\rangle=\left\langle\mathbf{y},\mathbf{x}\right\rangle$;

~~~~(3)线性性:$\left\langle\lambda\mathbf{x}+\mu\mathbf{y},z\right\rangle=\lambda\left\langle\mathbf{x},\mathbf{z}\right\rangle+\mu\left\langle\mathbf{y},\mathbf{z}\right\rangle$;

~~~~(4)柯西不等式$^*$:$\left\langle\mathbf{x},\mathbf{y}\right\rangle^2\leqslant \left\langle\mathbf{x},\mathbf{x}\right\rangle\cdot\left\langle\mathbf{y},\mathbf{y}\right\rangle$。

3. 距离的性质:

~~~~(1)正定性:$|\mathbf{x}-\mathbf{y}|\geqslant 0$,当且仅当$\mathbf{x}=\mathbf{y}$取等;

~~~~(2)对称性:$|\mathbf{x}-\mathbf{y}|=|\mathbf{y}-\mathbf{x}|$;

~~~~(3)三角不等式$^*$:$|\mathbf{x}-\mathbf{z}|\leqslant |\mathbf{x}-\mathbf{y}|+|\mathbf{y}-\mathbf{z}|$。

4. 邻域:设$\mathbf{a}=(a_1,\cdots,a_n)\in \mathbb{R}^n$,$\delta>0$,称点集
\begin{equation*}
    O(\mathbf{a},\delta)=\{x\in \mathbb{R}^n\Big| |\mathbf{x}-\mathbf{a}|<\delta\}
\end{equation*}
为点$\mathbf{a}$的$\delta$邻域,记作$O(\mathbf{a},\delta)$。

5. 点和$\mathbb{R}^n$上点集的关系:设$\mathbf{x}\in \mathbb{R}^n$,$S\subset \mathbb{R}^n$,

~~~~(1)内点:若存在$\delta>0$,使得$O(\mathbf{x},\delta)\subset S$,则称$\mathbf{x}$为$S$的内点;

~~~~(2)边界点:若$\mathbf{x}$的任意邻域$O(\mathbf{x},\delta)$内,都同时存在$S$内和$S$外的点,则称$x$为$S$的边界点;

~~~~(3)外点:若存在$\delta>0$,使得$O(\mathbf{x},\delta)\subset S^C$,则称$\mathbf{x}$为$S$的外点;

~~~~(4)聚点:若$\mathbf{x}$的任意邻域内都含有$S$中的无穷多个点,则称$\mathbf{x}$为$S$的聚点。

6. 开集和闭集;

~~~~(1)开集:若$S$中的每一个点都是$S$的内点,则称$S$为开集;

~~~~(2)闭集:若$S$的所有聚点均属于$S$,则称$S$为闭集。

7. 欧氏空间基本定理:

~~~~(1)闭矩形套定理:设$\Delta_k=[a_k,b_k]\times [c_k,d_k]$是$\mathbb{R}^2$上一列矩形,若$\Delta_{k+1}\subset \Delta_k$,且$\lim\limits_{k\rightarrow\infty}\sqrt{(b_k-a_k)^2+(d_k-c_k)^2}=0$,则
存在唯一的点$\mathbf{a}$在所有$\Delta_k$内,且该点的横坐标等于$\{a_k\}$和$\{b_k\}$的极限值,纵坐标等于$\{c_k\}$和$\{d_k\}$的极限值。

~~~~(2)Bolzano-Weierstrass定理:$\mathbb{R}^n$上的有界点列必有收敛子列。

~~~~(3)柯西收敛准则:$\mathbb{R}^n$上的点列$\{x_n\}$收敛的充要条件是:$\forall \varepsilon>0$,$\exists N \in \mathbb{N}^*$,$\forall m>n>N$,
$|\mathbf{x}_m-\mathbf{x}_n|<\varepsilon$。

\subsection{多元连续函数}

1. 多元函数:设$D$是$\mathbb{R}^n$上的点集,$D$到$\mathbb{R}$上的映射$f:D\rightarrow \mathbb{R}$称为$n$元函数。

2. 多元函数的$n$重极限:对任意$\varepsilon>0$,存在$\delta>0$,使得$\mathbf{x}_0$的去心邻域$O(\mathbf{x}_0,\delta)\backslash \{\mathbf{x}_0\}$
内的任一点$\mathbf{x}$,均有$|f(\mathbf{x})-A|<\varepsilon$,则称多元函数$f(\mathbf{x})$在$\mathbf{x}_0$处收敛于$A$,记作$\lim\limits_{\mathbf{x}\rightarrow\mathbf{x}_0}f(\mathbf{x})=A$。

3. 累次极限:对二元函数$f(x,y)$而言,若先求$x\rightarrow x_0$时$f(x,y)$的极限,再求$y\rightarrow y_0$时的极限,得到的结果称为先$x$后$y$的二次极限,
记作$\lim\limits_{y\rightarrow y_0}\lim\limits_{x\rightarrow x_0}f(x,y)$;同理可以定义先$y$后$x$的二次极限。

4. 二次极限和二重极限的关系:二次极限存在时,二重极限不一定存在;反之,二重极限存在时,二次极限一定存在且等于二重极限。

5. 多元函数连续性:若$\lim\limits_{\mathbf{x} \rightarrow x_0}f(\mathbf{x})=f(\mathbf{x}_0)$,则称$f(\mathbf{x})$在$\mathbf{x}_0$处连续;若$f(\mathbf{x})$在$D$上每一点均连续,则称
$f$是$D$上的连续函数。

6. 向量值函数:设$D$是$\mathbb{R}^n$上的点集,$D$到$\mathbb{R}^m$的映射$f:D\rightarrow \mathbb{R}^m$称为$n$元$m$维向量值函数。

7. 连续函数的性质:

~~~~(1)有界性定理:若$K$为$\mathbb{R}^n$上的有界闭集,$f$是$K$上的连续函数,则$f$在$K$上有界;

~~~~(2)最值定理:设$K$是$\mathbb{R}^n$上的有界闭集,$f$是$K$上的连续函数,则$f$在$K$上必定能取到最大值和最小值;

~~~~(3)一致连续:若$K$是$\mathbb{R}^n$中的点集,$f:K\rightarrow \mathbb{R}^n$为映射。若$\forall \varepsilon>0$,$\exists \delta >0$,$\forall \mathbf{x}_1,\mathbf{x}_2\in K$,$|\mathbf{x}_1-\mathbf{x}_2|<\delta$,
均有$|f(\mathbf{x}_1)-f(x_2)|<\varepsilon$,则称$f(\mathbf{x})$在$K$上一致连续。

~~~~(4)一致连续性定理:若$K$是$\mathbb{R}^n$上的有界闭集,$f:K\rightarrow \mathbb{R}^n$为连续映射,则$f$在$K$上一致连续。

\subsection{偏导数与全微分}

1. 定义:设$D\subset \mathbb{R}^2$为开集,$z=f(x,y)$为定义在$D$上的二元函数,$\left(x_0,y_0\right)\in D$为一定点。

~~~~(1)关于$x$的偏导:若极限
\begin{equation*}
    \lim\limits_{\Delta x\rightarrow 0}\frac{f(x_0+\Delta x,y_0)-f(x_0,y_0)}{\Delta x}
\end{equation*}
存在,则称$f$在$(x_0,y_0)$点关于$x$可偏导,极限值称为$f$关于$x$的偏导数。

~~~~(2)关于$y$的偏导:若极限
\begin{equation*}
    \lim\limits_{\Delta y\rightarrow 0}\frac{f(x_0,y_0+\Delta y)-f(x_0,y_0)}{\Delta y}
\end{equation*}
存在,则称$f$在$(x_0,y_0)$点关于$y$可偏导,极限值称为$f$关于$y$的偏导数。

~~~~(3)偏导函数:若$f$在任意$(x_0,y_0)\in D$处关于$x$均可偏导,则称$f$在$D$上可对$x$偏导,得到的偏导函数记作$\frac{\partial f }{\partial x}$或$f_x$;
同理,可将$f$关于$y$的偏导记作$\frac{\partial f}{\partial y}$或$f_y$。

2. 方向导数:设$z=f(x,y)$为定义在$D$上的二元函数,$\mathbf{v}=(\cos \alpha,\sin \alpha)$为一个方向。记
\begin{equation*}
    \frac{\partial f}{\partial\mathbf{v}}=f_x \cos\alpha +f_y \sin\alpha
\end{equation*}
为$f$沿方向$\mathbf{v}$的方向导数。

3. 全微分:若$z$的增量$\Delta z$可用$x$的增量、$y$的增量表示为
\begin{equation*}
    \Delta z =A \Delta x +B\Delta y+o\left(\sqrt{\Delta x^2+\Delta y^2}\right)
\end{equation*}
则称函数$z=f(x,y)$是可微的,并记线性主要部分$A\Delta x +B \Delta y$为$f(x,y)$的全微分,记作$\mathrm{d}z=A\mathrm{d}x+B\mathrm{d}y$。
其中$A(x,y)=f_x$,$B(x,y)=f_y$。

4. 梯度:称向量$(f_x,f_y)$为函数$z=f(x,y)$的梯度,记作\textbf{grad}$f$。梯度表示了函数值增加最快的方向。

5. 高阶偏导数:对$f$求偏导数后再求偏导数的结果称为二阶偏导数,对$f$求$n$次偏导的结果称为$n$阶偏导数。
二阶偏导数按照对$x$和$y$求偏导的次序分为四种:$f_{xx}$、$f_{xy}$、$f_{yx}$、$f_{yy}$,其中$f_{xy}=f_{yx}$。

6. 雅可比行列式:考虑向量值函数
\begin{equation*}
    f(\mathbf{x})=\left\{
        \begin{aligned}
            y_1 & = & f_1(x_1,\cdots,x_n)\\
            y_2 & = & f_2(x_2,\cdots,x_n)\\
        &\vdots&\\
            y_m & = & f_m(x_2,\cdots,x_n)
        \end{aligned}\right.
\end{equation*}
在每一个坐标分量$y_i$处,对每一个自变量$x_j$求偏导数,将结果用矩阵
\begin{equation*}
    J=\begin{pmatrix}
        \frac{\partial f_1}{\partial x_1} &\cdots&\frac{\partial f_1}{\partial x_n}\\
        \vdots&\vdots&\vdots\\
        \frac{\partial f_m}{\partial x_1} &\cdots&\frac{\partial f_m}{\partial x_n}
    \end{pmatrix}
\end{equation*}
表示,该矩阵称为向量值函数的雅可比矩阵,记作$f'(\mathbf{x})$或$\mathrm{D}f$。雅可比矩阵相当于向量值函数的导数。

7. 连续、可偏导和可微的关系:可微必连续;可微必可偏导;可偏导不一定可微,但偏导数连续时一定可微。

\subsection{多元函数求导法则}

1. 一般求导方法:对$x$求偏导时,把$y$视作常量,仅对$x$求导;对$y$求偏导时,把$x$视作常量,仅对$y$求导。

2. 多元复合函数的链式求导法则:设$z=z(x,y)$,其中$x=x(u,v)$,$y=y(u,v)$则有:
\begin{eqnarray*}
    z_u=z_x\cdot x_u+z_y \cdot y_u
    z_v = z_x \cdot x_v +z_y \cdot y_v
\end{eqnarray*}
面对更为复杂的情况时,首先按照复合次序从上到下列出树状图,再考虑自顶向下所有能连到目标变量的路径,求处此路径上所有偏导数的乘积,最终将它们求和。

3. 一阶全微分的形式不变性:无论$x$、$y$是自变量还是中间变量,一阶全微分$\mathrm{d}z=z_x\mathrm{d}x+z_y\mathrm{d}y$这一形式始终不变。

4. 隐函数存在性定理:若二元函数$F(x,y)$满足$F(x_0,y_0)=0$,在以$(x_0,y_0)$为中心的某一闭矩形上连续且具有连续偏导数,同时$F_y(x_0,y_0)\neq 0$,则
可以从隐函数$F(x,y)=0$中唯一确定隐函数$y=f(x)$。

5. 多元隐函数求导方法:对$F(y,x_1,x_2,\cdots,x_n)$两边分别关于$x_1,x_2,\cdots,x_n$求导,将$\frac{\partial y}{\partial x_1},\cdots,\frac{\partial y}{\partial x_n}$视作变量,求解$n$元方程组。

\subsection{多元函数微分学的应用}

1. 中值定理:设$f(x,y)$在凸区域$D\subset \mathbb{R}^2$上可微,则对$D$上任意两点$(x_0,y_0)$和$x_0+\Delta x,y_0+\Delta y$,存在$0<\theta<1$,记$\mathbf{t }= (x_0+\theta \Delta x,y_0+\theta \Delta y)$,则
\begin{equation*}
    f(x_0+\Delta x,y_0+\Delta y)-f(x_0,y_0)=f_x(\mathbf{t})\Delta x+f_y(\mathbf{t})\Delta y
\end{equation*}

2. 泰勒展开:设$f(x,y)$在点$(x_0,y_0)$的邻域$U$上具有$n+1$阶连续偏导数,则$U$内每一点成立
\begin{equation*}
    f(x_0+\Delta x,y_0+\Delta y)=\sum\limits_{k=0}^n \frac{1}{k!}\left(\frac{\partial}{\partial x}\Delta x +\frac{\partial}{\partial y}\Delta y\right)^k f(x_0,y_0)+o\left(\left(\Delta x^2+\Delta y^2\right)^\frac{k}{2}\right)
\end{equation*}

3. 空间曲线的切线和法平面:

~~~~(1)空间曲线的参数方程:设参数$t\in[a,b]$,则参数方程
\begin{equation*}
    \Gamma:\left\{\begin{aligned}
        x=x(t)\\
        y=y(t)\\
        z=z(t)
    \end{aligned}\right.
\end{equation*}
表示空间中的一条曲线。

~~~~(2)空间曲线的切向量:向量$\mathbf{r}'(t_0)=\left(x'(t_0),y'(t_0),z'(t_0)\right)$是曲线$\Gamma$在$t=t_0$处的切向量。

~~~~(3)空间曲线的切线方程:当$t=t_0$时,空间曲线$\Gamma$的切线方程为
\begin{equation*}
    \frac{x-x_0}{x'(t_0)}=\frac{y-y_0}{y'(t_0)}=\frac{z-z_0}{z'(t_0)}
\end{equation*}

~~~~(4)空间曲线的法平面:过定点$\left(x(t_0),y(t_0),z(t_0)\right)$且与该点切线垂直的平面,方程为
\begin{equation*}
    x'(t_0)(x-x_0)+y'(t_0)(y-y_0)+z'(t_0)(z-z_0)=0
\end{equation*}

4. 曲面的切平面和法线:

~~~~(1)空间曲面方程:$S:F(x,y,z)=0$。

~~~~(2)空间曲面的切平面:在$P_0(x_0,y_0,z_0)$处,空间曲面$S$的切平面方程为
\begin{equation*}
    F_x(P_0)(x-x_0)+F_y(P_0)(y-y_0)+F_z(P_0)(z-z_0)=0
\end{equation*}

~~~~(3)空间曲面的法向量:$\mathbf{n}=\left(F_x(P_0),F_y(P_0),F_z(P_0)\right)$。

~~~~(4)空间曲面的法线:
\begin{equation*}
    \frac{x-x_0}{F_x(P_0)}=\frac{y-y_0}{F_y(P_0)}=\frac{z-z_0}{F_z(P_0)}
\end{equation*}

6. 求函数最值:设nabla算子$\nabla =\left(\frac{\partial}{\partial x},\frac{\partial}{\partial y}\right)$,
则求函数$f(x,y)$的最值时,先求解$\nabla f = \mathbf{0}$得到$f$的所有极值,再比较得出$f$的最大值和最小值。


7. 条件极值:设函数$f=f(x_1,x_2,\cdots,x_n)$满足约束$G_i(x_1,\cdots,x_n)=0$,其中$i=1,2,\cdots,m$,构建Lagrange函数
\begin{equation*}
    L(x_1,\cdots,x_n,\lambda_1,\cdots,\lambda_m)=f(x_1,\cdots,x_n)+\sum\limits_{i=1}^m \lambda_iG_i(x_1,\cdots,x_n)
\end{equation*}

列出$m+n$个方程:$\frac{\partial L}{\partial x_i}=0$与$\frac{\partial L}{\partial \lambda_i}=0$,从中解出极值点$(x_1,\cdots,x_n)$满足的条件。此方法称为Lagrange乘数法。

\section{多元函数积分学}

\begin{tcolorbox}[colback=red!5,colframe=red!75!black]
    ~~~~和之前的一元函数微积分略有区别,学习多元函数积分学的重心不在于理论证明,而更注重实际应用。
   多元函数积分对现实问题的刻画更为深刻,每一种线面积分都有实际的物理意义,在物理、工程上有非常广泛的应用。

    ~~~~当积分区域从平面变为空间时,首先要解决的是最简单的问题,即在$xOy$平面的投影是矩形的情况,这就引入了二重积分。若为空间
    里的每个点赋予密度,对空间几何体求质量的问题就可转化为三重积分。因此,学习多元函数积分学的第一步,是掌握重积分的各种技巧。

    ~~~~若积分对象变得不规则,如空间中的任意曲线、曲面,此时需要引入另一类积分,那就是线面积分。若积分时不区分方向,则称为对弧长/面积的线面积分,即
    第一类线面积分;若积分时考虑方向,则称为对坐标的线面积分,即第二类线面积分。作为物理中的重要应用,还介绍了部分关于场论的概念。

    ~~~~在数学分析中有一个非常优美的公式——流形上的Stokes公式,它统一了微积分基本定理、格林公式、高斯公式和斯托克斯公式。为了介绍该著名公式,额外引入了微分形式与外微分作为铺垫。

    ~~~~最后介绍了含参变量积分。若对二元函数中的某一个变量积分,得到的便是关于另一个变量的函数,这种函数被称为含参变量积分。它是数学分析中某些后续课程的基础,因此有必要介绍含参变量积分的相关性质。
\end{tcolorbox}

\subsection{重积分}

1. 二重积分:考虑一个曲顶柱体,底面是$xOy$平面上的有界闭区域$D$,顶面是非负连续函数$z=f(x,y)$,则该曲顶柱体的体积便是$f(x,y)$在区域$D$上的二重积分,记作
\begin{equation*}
    \iint\limits_D f(x,y)\mathrm{d} \sigma
\end{equation*}
其中$\mathrm{d}\sigma$为面积微元。

2. 二重积分的计算:

~~~~(1)直角坐标系上:$\mathrm{d}\sigma$可用$\mathrm{d}x\mathrm{d}y$表示。

~~~~(2)先$y$后$x$积分法:若$D$的横坐标取值范围是$[a,b]$,对任意$x_0\in [a,b]$,$D$与$x=x_0$所交线段纵坐标在$[f(x_0),g(x_0)]$内,则
\begin{equation*}
    \iint\limits_{D}f(x,y)\mathrm{d}x\mathrm{d}y=\int_a^b\mathrm{d}x\left[\int_{f(x)}^{g(x)}f(x,y)\mathrm{d} y\right]
\end{equation*}

~~~~(3)先$x$后$y$积分法:若$D$的纵坐标取值范围是$[c,d]$,对任意$y_0\in [c,d]$,$D$与$y=y_0$所交线段横坐标在$[f(y_0),g(y_0)]$内,则
\begin{equation*}
    \iint\limits_{D}f(x,y)\mathrm{d}x\mathrm{d}y=\int_c^d\mathrm{d}y\left[\int_{f(y)}^{g(y)}f(x,y)\mathrm{d} x\right]
\end{equation*}

3. 二重积分的变量代换:

~~~~(1)极坐标代换公式:设$x=r\cos\theta$,$y=r\sin\theta$,其中$\theta\in [\theta_1,\theta_2]$,$r$的取值范围为$[r_1(\theta),r_2(\theta)]$,则
\begin{equation*}
    \iint\limits_D f(x,y)\mathrm{d}x\mathrm{d}y=\int_{\theta_1}^{\theta_2} \mathrm{d}\theta \left[\int_{r_1(\theta)}^{r_2(\theta)}f(x,y)\cdot r\mathrm{d}r\right]
\end{equation*}

~~~~(2)一般坐标代换公式:设$x=x(u,v)$,$y=y(u,v)$,雅可比行列式
\begin{equation*}
    \frac{\partial(x,y)}{\partial(u,v)}=\left|\begin{matrix}
        x_u&x_v\\
        y_u&y_v
    \end{matrix}\right|
\end{equation*}
则面积微元可按照如下公式代换:
\begin{equation*}
    \mathrm{d}x\mathrm{d}y=\left|\frac{\partial(x,y)}{\partial(u,v)}\right|\mathrm{d}u\mathrm{d}v
\end{equation*}

4. 三重积分:考虑空间上的封闭几何体$\Omega$,每一点$(x,y,z)$处的密度为$f(x,y,z)$,则该几何体的质量便是$f(x,y,z)$在空间几何体$\Omega$上的三重积分,记作
\begin{equation*}
    \iiint\limits_\Omega f(x,y,z)\mathrm{d}V
\end{equation*}

5. 三重积分的计算:

~~~~(1)直角坐标系上:$\mathrm{d}V$可用$\mathrm{d}x\mathrm{d}y\mathrm{d}z$表示。

~~~~(2)先1后2积分法:先求空间几何体在平面上的投影$D$(以$xOy$平面为例),过$D$内任一点$(x,y)$作$z$轴平行线,与空间几何体所交线段的$z$坐标范围为$[z_1,z_2]$,则
\begin{equation*}
    \iiint\limits_\Omega f(x,y,z)\mathrm{d}x \mathrm{d}y\mathrm{d}z = \iint\limits_D \mathrm{d}x\mathrm{d}y\left[\int_{z_1}^{z_2} f(x,y,z)\mathrm{d}z\right]
\end{equation*}

~~~~(3)先2后1积分法:设空间几何体的$z$坐标范围为$[z_1,z_2]$,过任意$z\in[z_1,z_2]$作平行于$xOy$的平面,交空间几何体于平面区域$D(z)$,则
\begin{equation*}
    \iiint\limits_\Omega f(x,y,z)\mathrm{d}x\mathrm{d}y\mathrm{d}z=\int_{z_1}^{z_2}\mathrm{d}z\left[\iint\limits_{D(z)}f(x,y,z)\mathrm{d}x\mathrm{d}y\right]
\end{equation*}

6. 三重积分的变量代换:

~~~~(1)柱坐标代换公式:令$x=r\cos\theta$,$y=r\sin\theta$,$z$保持不变,则
\begin{equation*}
    \iiint\limits_{\Omega}f(x,y,z)\mathrm{d}x\mathrm{d}y\mathrm{d}z=\iiint\limits_{\Omega}f(r,\theta,z)r\mathrm{d}r\mathrm{d}\theta\mathrm{d}z
\end{equation*}

~~~~(2)球坐标代换公式:令$x=r\sin\varphi\cos\theta$,$y=r\sin\varphi\sin\theta$,$z=r\cos\varphi$,则
\begin{equation*}
    \iiint\limits_{\Omega}f(x,y,z)\mathrm{d}x\mathrm{d}y\mathrm{d}z=\iiint\limits_{\Omega}f(r,\theta,\varphi)r^2\sin\varphi\mathrm{d}r\mathrm{d}\theta\mathrm{d}\varphi
\end{equation*}
其中$\varphi$表示和$z$轴负半轴的夹角范围,通常取$[0,\pi]$;$\theta$表示$xOy$平面上旋转角度,通常取$[0,2\pi]$。

~~~~(3)一般坐标代换公式:令$x,y,z$分别为$u,v,w$的函数,则
\begin{equation*}
    \iiint\limits_{\Omega}f(x,y,z)\mathrm{d}x\mathrm{d}y\mathrm{d}z=\iiint\limits_{\Omega}f(u,v,w)\left|\frac{\partial(x,y,z)}{\partial(u,v,w)}\right|\mathrm{d}u\mathrm{d}v\mathrm{d}w
\end{equation*}

\subsection{曲线积分}

1. 第一类曲线积分:对弧长的曲线积分。考虑一段各点密度$f(x,y)$已知的平面曲线$L$,对其求质量时,取弧长微元$\mathrm{d}s$,则有第一类曲线积分:
\begin{equation*}
    \int\limits_L f(x,y)\mathrm{d} s
\end{equation*}
当$L$为空间曲线时,弧长微元仍为$\mathrm{d}s$,第一类曲线积分定义为$\int \limits_L f(x,y,z)\mathrm{d} s$。

2. 第一类曲线积分求法:平面曲线有$\mathrm{d}s=\sqrt{(x')^2+(y')^2} \mathrm{d}x\mathrm{d}y$;空间曲线有$\mathrm{d}s=\sqrt{(x')^2+(y')^2+(z')^2} \mathrm{d}x\mathrm{d}y\mathrm{d}z$。

3. 第二类曲线积分;对坐标的曲线积分。考虑一段标定方向的空间曲线$L$,以力$f(x,y)=\left(P(x,y),Q(x,y)\right)$沿着$L$做功,则有第二类曲线积分:
\begin{equation*}
    \int\limits_L P(x,y)\mathrm{d}x+Q(x,y)\mathrm{d}y
\end{equation*}
空间曲线上的第二类曲线积分定义为$\int\limits_L P(x,y,z)\mathrm{d}x+Q(x,y,z)\mathrm{d}y+R(x,y,z)\mathrm{d}z$。

4. 第二类曲线积分求法:用参数$t$描述曲线$L$,则可以代入$\mathrm{d}x=x'(t)\mathrm{d}t$,同理有$\mathrm{d}y=y'(t)\mathrm{d}t$和$\mathrm{d}z=z'(t)\mathrm{d}t$。

5. 格林公式:设$D$为平面上光滑或分段光滑的简单闭曲线所围成的单连通闭区域,$\partial D$为逆时针方向的区域边界,且函数$P(x,y)$和$Q(x,y)$在$D$上具有连续偏导数,则有
\begin{equation*}
    \oint\limits_{\partial D}P\mathrm{d}x+Q\mathrm{d}y=\iint\limits_{D}\left(\frac{\partial Q}{\partial x}-\frac{\partial P}{\partial y}\right)\mathrm{d}x\mathrm{d}y
\end{equation*}

6. 曲线积分与路径无关的条件:若$P \mathrm{d}x +Q\mathrm{d}y$恰好为函数$u(x,y)$的全微分,即
\begin{equation*}
    \mathrm{d}u=P\mathrm{d}x+Q\mathrm{d}y
\end{equation*}
则曲线积分$\int\limits_{L}P\mathrm{d}x+Q\mathrm{d}y$与路径$L$无关,仅与起点$A$和终点$B$有关,且积分值为$u(x,y)\big|_A^B$。
判定方式:$\frac{\partial P}{\partial y}=\frac{\partial Q}{\partial x}$。

\subsection{曲面积分}

1. 第一类曲面积分:对面积的曲面积分。考虑各点密度$f(x,y,z)$已知的空间曲面$\Sigma$,对其求质量时,取面积微元$\mathrm{d}S$,则有第一类曲面积分:
\begin{equation*}
    \iint\limits_{\Omega}f(x,y,z)\mathrm{d}S
\end{equation*}

2. 第一类曲面积分求法:当曲面方程可用$z=z(x,y)$,$(x,y)\in D$表示时,面积微元$\mathrm{d}S =\sqrt{1+(z_x)^2+(z_y)^2}\mathrm{d}x\mathrm{d}y$,即
\begin{equation*}
    \iint\limits_{\Omega}f(x,y,z)\mathrm{d}S=\iint\limits_D f\left(x,y,z(x,y)\right)\sqrt{1+(z_x)^2+(z_y)^2}\mathrm{d}x\mathrm{d}y
\end{equation*}

3. 第二类曲面积分:对坐标的曲面积分。考虑各点流速$\left(P,Q,R\right)$已知的流体在定向曲面$\Sigma$上的流量,则有第二类曲面积分:
\begin{equation*}
    \iint\limits_{\Omega}P(x,y,z) \mathrm{d}y\mathrm{d}z+Q(x,y,z) \mathrm{d}z\mathrm{d}x+R(x,y,z) \mathrm{d}x\mathrm{d}y
\end{equation*}

4. 第二类曲面积分求法:首先确定$\Omega$定向。以$P\mathrm{d}x\mathrm{d}y$分量为例,用平行于$z$轴正方向的线穿过曲面,若该线穿入曲面则定为反向(二重积分变号),穿出曲面则定为正向。
定向后,分别将$P\mathrm{d}x\mathrm{d}y$、$Q\mathrm{d}y\mathrm{d}z$、$R\mathrm{d}z\mathrm{d}x$投影到$xOy$、$yOz$、$zOx$平面,分别在各投影面上直接用二重积分的方法求解各个分量的积分值,最后求和。
此方法麻烦且不常用,一般都可以用后面介绍的高斯公式求解。

5. 高斯公式:设$\Omega$是由光滑或分片光滑的封闭曲面围成的闭区域,函数$P(x,y,z)$,$Q(x,y,z)$,$Q(x,y,z)$在$\Omega$上具有连续偏导数,$\partial \Omega$为$\Omega$的外侧曲面,则
\begin{equation*}
    \oiint\limits_{\partial \Omega}P\mathrm{d}y\mathrm{d}z+Q\mathrm{d}z\mathrm{d}x+R\mathrm{d}x\mathrm{d}y=\iiint\limits_\Omega \left(\frac{\partial P}{\partial x}+\frac{\partial Q}{\partial y}+\frac{\partial R}{\partial z}\right)\mathrm{d}x\mathrm{d}y\mathrm{d}z
\end{equation*}

6. 斯托克斯公式:设$\Sigma$为光滑曲面,其边界$\partial \Sigma$为分段光滑闭曲线。若函数$P(x,y,z)$,$Q(x,y,z)$,$R(x,y,z)$在$\Sigma$以及$\partial \Sigma$上具有连续偏导数,则成立
\begin{equation*}
    \int\limits_{\partial \Sigma} P\mathrm{d}x+Q\mathrm{d}y+R\mathrm{d}z=\iint\limits_\Sigma \left|\begin{matrix}
        \mathrm{d}y\mathrm{d}z &\mathrm{d}z\mathrm{d}x&\mathrm{d}x\mathrm{d}y\\
        \frac{\partial }{\partial x}&\frac{\partial }{\partial y}&\frac{\partial }{\partial z}\\
        P&Q&R\\
    \end{matrix}\right|
\end{equation*}

\subsection{流形上的Stokes公式}

1. 向量外积:设$\mathbf{a}=(a_1,a_2)$,$\mathbf{b}=(b_1,b_2)$,则定义向量外积
\begin{equation*}
    a \wedge b = \left|\begin{matrix}
        a_1&a_2\\
        b_1&b_2
    \end{matrix}\right|
\end{equation*}
外积具有反对称性和对加法的分配律,且$\mathbf{a}\wedge \mathbf{a}=0$。向量外积的几何意义为平行四边形的有向面积。

2.微分形式:以微分及其外积作为一组基的向量空间。

~~~~(1)一次微分形式:令$\mathbf{x}=(x_1,\cdots,x_n)$,则$1-$形式为
\begin{equation*}
    a_1(\mathbf{x})\mathrm{d}x_1 +\cdots +a_n(\mathbf{x})\mathrm{d}x_n \in \Lambda^1
\end{equation*}

~~~~(2)二次微分形式:由反对称性得$\mathrm{d}x_i \wedge \mathrm{d}x_j =-\mathrm{d}x_j \wedge \mathrm{d}x_i$,则$2-$形式为
\begin{equation*}
    \sum\limits_{1\leqslant i<j \leqslant n} g_{ij}(\mathbf{x}) \mathrm{d}x_i \wedge \mathrm{d}x_j \in \Lambda^2
\end{equation*}

3. 微分形式的外积:令$\omega = a_1(\mathbf{x})\mathrm{d}x_1+\cdots + a_n(\mathbf{x})\mathrm{d}x_n$,$\eta = b_1(\mathbf{x})\mathrm{d}x_1+\cdots + b_n(\mathbf{x})\mathrm{d}x_n$:
\begin{align*}
    \omega \wedge \eta &=\sum\limits_{i,j=1}^n a_i(\mathbf{x})b_j(\mathbf{x})\mathrm{d}x_i \wedge \mathrm{d}x_j\\
    &=\sum\limits_{1\leqslant i<j \leqslant n}\left|\begin{matrix}
        a_i(\mathbf{x}) & a_j(\mathbf{x})\\
        b_i(\mathbf{x}) & b_j(\mathbf{x})
    \end{matrix}\right| \mathrm{d}x_i \wedge \mathrm{d}x_j
\end{align*}

4. 外微分:对$\Lambda^1$的任意$1-$形式$\omega = a_1(\mathbf{x})\mathrm{d}x_1+\cdots+a_n(\mathbf{x})\mathrm{d}x_n$,
定义$\mathrm{d} \omega = \mathrm{d}a_1(\mathbf{x})\wedge \mathrm{d}x_1+\cdots+\mathrm{d}a_n(\mathbf{x})\wedge \mathrm{d}x_n$。
同理,当$\omega$为$2-$形式时,定义
\begin{equation*}
    \mathrm{d}\omega = \sum\limits_{1\leqslant i<j \leqslant n} \mathrm{d}g_{ij}(\mathbf{x})\wedge  \mathrm{d}x_i \wedge \mathrm{d}x_j
\end{equation*}

5. 流形上的斯托克斯公式:高次微分形式$\mathrm{d}\omega$在给定区域上的积分等于低一次的微分形式$\omega$ 在低一维的区域边界上的积分。定义$M$为微分流形,$\partial M$为$M$具有诱导定向的边界,则
\begin{equation*}
    \int \limits_{\partial M}\omega = \int\limits_{M}\mathrm{d}\omega
\end{equation*}

~~~~(1)Newton-Leibniz公式:令$M=[a,b]$,则$\partial M=\big|_a^b$。此时
\begin{equation*}
    F(x)\big|_a^b =\int_a^b \mathrm{d}F(x)=\int_a^b f(x)\mathrm{d}x
\end{equation*}

~~~~(2)Green公式:令$M$为平面封闭区域$D$,取诱导定向为逆时针,此时
\begin{align*}
    \oint\limits_{\partial D}P\mathrm{d}x+Q\mathrm{d}y &=\iint\limits_D \left(P_x\mathrm{d}x+P_y\mathrm{d}y\right)\wedge\mathrm{d}x+\left(Q_x\mathrm{d}x+Q_y\mathrm{d}y\right)\wedge\mathrm{d}y\\
&=\iint\limits_D \left(Q_y-P_x\right)\mathrm{d}x\wedge \mathrm{d}y
\end{align*}


~~~~(3)Gauss公式:令$M$为空间封闭区域$\Omega$,取诱导定向为外侧,此时
\begin{align*}
    &\oiint\limits_{\partial \Omega} P\mathrm{d} y\wedge\mathrm{d}z+Q\mathrm{d}z\wedge\mathrm{d}x+R\mathrm{d}x\wedge\mathrm{d}y\\
    &=\iiint\limits_{\Omega}\left(P_x\mathrm{d}x\right)\wedge \mathrm{d}y \wedge \mathrm{d}z+\left(Q_y\mathrm{d}y\right)\wedge \mathrm{d}z \wedge \mathrm{d}x+\left(R_z\mathrm{d}z\right)\wedge \mathrm{d}x \wedge \mathrm{d}y\\
    &=\iiint\limits_{\Omega}\left(P_x+Q_y+R_z\right)\mathrm{d}x\wedge\mathrm{d}y\wedge\mathrm{d}z
\end{align*}

~~~~(4)Stokes公式:令$M$为空间曲面$S$,取诱导定向为逆时针,此时
\begin{align*}
    &\oint\limits_{\partial D}P\mathrm{d}x+Q\mathrm{d}y+R\mathrm{d}z=\iint\limits_{D}\left(P_y\wedge \mathrm{d} y +P_z \wedge \mathrm{d} z\right)\wedge \mathrm{d}x \\
    &+\left(Q_x\wedge \mathrm{d} x +Q_z \wedge \mathrm{d} z\right)\wedge \mathrm{d}y+\left(Q_x\wedge \mathrm{d} x +R_y \wedge \mathrm{d} y\right)\wedge \mathrm{d}z\\
    &=\iint\limits_{D} \left|\begin{matrix}
        \mathrm{d}y\wedge\mathrm{d}z &\mathrm{d}z\wedge\mathrm{d}x&\mathrm{d}x\wedge\mathrm{d}y\\
        \frac{\partial }{\partial x} &\frac{\partial }{\partial y}&\frac{\partial }{\partial z}\\
        P&Q&R  
    \end{matrix}\right|
\end{align*}

\subsection{场论}

1. 场:设$\Omega\subset \mathbb{R}^3$为一个区域,若在$t$时刻,$\Omega$中每一点$(x,y,z)$都有一个确定的$f(x,y,z,t)$与之对应,
则称$f$为$\Omega$上的场。当$f$为多元函数时,称为数量场;当$f$为向量值函数时,称为向量场。

2. 梯度:若$f(x,y,z)$在$\Omega$上具有连续偏导数,则定义$f$的梯度为
\begin{equation*}
    \mathbf{grad} f=f_x \mathbf{i}+f_y \mathbf{j}+f_z\mathbf{k}
\end{equation*}
函数沿梯度方向上升最快。

3. 散度:设$\mathbf{a}=(P,Q,R)$为$\Omega$上的向量场,$M$为$\Omega$中一点,定义
\begin{equation*}
    \text{div } \mathbf{a}(M)=P_x(M)+Q_y(M)+R_z(M)
\end{equation*}
为向量场$\mathbf{a}$在$M$点的散度。当散度为正时,$M$为源点;当散度为负时,$M$为汇点;若$\Omega$中任一点的散度均为0,则称$\Omega$为无源场。

4. 旋度:设$\mathbf{a}=(P,Q,R)$为$\Omega$上的向量场,$M$为$\Omega$中一点,定义
\begin{equation*}
    \textbf{rot } \mathbf{a}(M)=\left|\begin{matrix}
        \mathbf{i} &\mathbf{j}&\mathbf{k}\\
        \frac{\partial }{\partial x} &\frac{\partial }{\partial y}&\frac{\partial }{\partial z}\\
        P(M)&Q(M)&R(M)  
    \end{matrix}\right|
\end{equation*}
为向量场$\mathbf{a}$在$M$点的旋度。当旋度处处为\textbf{0}时,$\textbf{a}$称为保守场,此时$\textbf{a}$中的曲线积分与路径无关。

5. Hamilton算子:

~~~~(1)Nabla算子:定义如下,且$\nabla f=\textbf{grad }f $,$\nabla \cdot \mathbf{a}=\text{div }\mathbf{a} $,$\nabla \times \mathbf{a}=\textbf{rot }\mathbf{a} $。
\begin{equation*}
    \nabla = \mathbf{i}\frac{\partial }{\partial x}+\mathbf{j}\frac{\partial }{\partial y}+\mathbf{k}\frac{\partial }{\partial z}
\end{equation*}

~~~~(2)Laplace算子:定义如下,且$\Delta=\nabla \cdot \nabla$,称$\Delta u=0$为调和方程。
\begin{equation*}
    \Delta = \frac{\partial^2}{\partial x^2}+\frac{\partial^2}{\partial y^2}+\frac{\partial^2}{\partial z^2}
\end{equation*}

\subsection{含参变量积分}

1. 定义:设$f(x)$是定义在闭矩形$[a,b]\times [c,d]$上的连续函数,则定义关于$y$的函数
\begin{equation*}
    I(y)=\int_a^b f(x,y)\mathrm{d}x
\end{equation*}
定义域为$[c,d]$。同样可以定义关于$x$的函数
\begin{equation*}
    J(x)=\int_c^d f(x,y)\mathrm{d}y
\end{equation*}
定义域$[a,b]$。这种对$f(x,y)$中某一个变量积分得到的函数称为含参变量积分。

2. 含参变量积分的分析性质:若$f(x)$在闭矩形上是连续函数,则积分得到的函数连续,积分号可分别与求导、极限、积分交换次序。

3. 含参变量积分求导:设$F(y)=\int_{a(y)}^{b(y)}f(x,y)\mathrm{d}x$,则
\begin{equation*}
    F'(y)=\int_{a(y)}^{b(y)}f_y(x,y)\mathrm{d}x+b'(y)f(b,y)-a'(y)f(a,y)
\end{equation*}

4. 含参变量广义积分:

~~~~(1)一致收敛定义:设$f(x,y)$的定义域为$[a,+\infty)\times[c,d]$。若$\forall \varepsilon>0$,$\exists A_0>0$,$\forall A>A_0$,$\forall y\in [c,d]$, 
\begin{equation*}
    \left|\int_a^A f(x,y)\mathrm{d} x-I(y)\right|<\varepsilon
\end{equation*}
则称$\int_a^{+\infty} f(x,y)\mathrm{d}x$关于$y$在$[c,d]$上一致收敛于$I(y)$。

~~~~(2)柯西收敛准则:$\int_a^{+\infty} f(x,y)\mathrm{d}x$关于$y$在$[c,d]$上一致收敛的充要条件:$\forall \varepsilon>0$,$\exists A_0>0$,$\forall A_2>A_1>A_0$,$\forall y\in[c,d]$,
\begin{equation*}
    \left|\int_{A_1}^{A_2}f(x,y)\mathrm{d}x\right|<\varepsilon
\end{equation*}

~~~~(3)Weierstrass判别法:若存在$F(x)$使得$|f(x,y)|\leqslant F(x)$在定义域内恒成立,且$\int_a^{+\infty}F(x)\mathrm{d}x$收敛,
则$\int_a^{+\infty} f(x,y)\mathrm{d}x$在$[c,d]$上一致收敛。

~~~~(4)A-D判别法:若下面某条件满足,则含参变量积分$\int_a^{+\infty}f(x,y)g(x,y)\mathrm{d}x$在$[c,d]$上一致收敛:

~~~~~~~~~~~~$\cdot$ Abel条件:$\int_a^{+\infty}f(x,y)\mathrm{d}x$在$[c,d]$上一致收敛,且对任意$y\in [c,d]$,$g(x,y)$关于$x$单调且一致有界;

~~~~~~~~~~~~$\cdot$ Dirichlet条件:$\int_a^{A}f(x,y)\mathrm{d}x$在$[c,d]$上关于$A$一致有界,且对任意$y \in [c,d]$,$g(x,y)$关于$x$单调且一致趋于$0$。
