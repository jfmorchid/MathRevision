\chapter{数学分析}
\thispagestyle{empty}

\setlength{\fboxrule}{0pt}\setlength{\fboxsep}{0cm}
\noindent\shadowbox{
\begin{tcolorbox}[arc=0mm,colback=lightblue,colframe=darkblue,title=Mathematical Analysis]
\kai{~~~~数学分析是大一新生所修的重要学科基础课, 相比非数学专业更强调证明, 对收敛性的讨论篇幅较大, 与大二的实变函数课程联系紧密. 数分
是今后多门专业课的先修课程: 积分学应用于概率论对随机变量的研究; 对积分的进一步研究(Lebesgue积分)是实变函数的重要内容; Fourier变换和多元函数积分学是偏微分方程必不可少的工具... 
山大主选教材为陈纪修的《数学分析》(第三版), 在此基础上结合卓里奇的数学分析教程, 对共计三个学期的数分课程进行完整的内容回顾.}\\

\kai{~~~~数分1的重点: 极限与连续概念, 一元函数微分学, 微分中值定理}\\

\kai{~~~~数分2的重点: 一元函数积分学, 数项级数和函数项级数, 广义积分}\\

\kai{~~~~数分3的重点: 多元函数微分学, 含参变量积分, 多元函数积分学(重积分, 曲线与曲面积分)}

\end{tcolorbox}}
\setlength{\fboxrule}{1pt}\setlength{\fboxsep}{4pt}


\newpage

\section{集合与函数概念}

\begin{tcolorbox}[colback=red!5,colframe=red!75!black]
    ~~~~这部分内容不多且不难, 可以认为是从高中数学到数分的过渡, 更可以认为是数学各个分支的基石.
    
    ~~~~集合论是高等数学的核心, 由此衍生出基(tu)础(tou)数学和计算机科学的区别: 一个研究连续集合, 比如实数域, 复数域等具有连续势集合上的映射, 另一个更偏向
    离散集合, 也就是有穷集和可列集上的映射. 从前者开始诞生实分析, 复分析, 傅里叶分析, 泛函分析等各大分析, 后者则衍生出图论, 组合数学, 数据结构等计算机科学分支. 认清这一点后, 
    我们便可以用一句话概括数学分析干了啥: 研究实数域或$n$维欧氏空间到实数域上的映射. 同时, 集合论又是各大学科的基础(笔者在数分, 实变, 离散数学三门课上过三遍集合论...), 故不可轻敌.

    ~~~~映射就是数学分析的研究主体. 注意到我们只研究欧氏空间到实数域的映射, 也就是实变量函数, 我们可以归纳出这一类函数的表示方法和基本性质, 同时温习一下高中数学内容.
    
\end{tcolorbox}

\subsection{集合论}

1. 集合的定义: 具有特定性质的一些对象总体, 集合内的对象称为元素.

\begin{tcolorbox}[colback=blue!5,colframe=blue!75!black,title=定义解析]
    ~~~~这一句话概括了集合的很多隐含属性: 首先,元素有"特定"的性质, 那么该性质直接决定
    某一对象是否属于集合, 即对象完全由性质确定; 第二, "总体"告诉我们, 研究集合要从整体上把握, 切勿以偏概全, 给集合下的结论应对集合内任一元素成立; 
    最后, 集合表示了对象是否具有某性质, 那么同一对象不会多次出现在一个集合内.
\end{tcolorbox}

2. 集合关系: (1) 从属关系: 若对象$x$在集合$A$内, 则称$x$属于$A$, 记作$x \in A$.

(2) 包含关系: 若$\forall x \in A, x \in B$, 则称$B$包含$A$, 记作$A \subset B$. 

3. 集合运算: (1) 交运算: $A \cap B=\{x|x\in A, x\in B\}$

(2) 并运算: $A \cup B=\{x|x\in A \text{~~or~~} x\in B\}$

(3) 补运算: $\overline{A}=\{x|x \notin A \}$

(4) De Morgan定律: $\overline{A \cap B}=\overline{A}\cup \overline{B};$ $\overline{A \cup B}=\overline{A}\cap \overline{B}.$

\begin{theo}{德摩根定律}{theo1.1}
只证前一部分, 后一部分证法类似.

对$\forall x \in \overline{A \cap B}$, $x \notin A \cap B$, 即$x \notin A$ 或$x \notin B$. 若$x \notin A$, 则$x \in \overline{A}$; 若$x \notin B$, 则$x \in \overline{B}$.
总之, 必有$x \in \overline{A}\cup \overline{B}.$

反之, 对$\forall x \in \overline{A}\cup \overline{B}$, $x \notin A$ 或$x \notin B$, 即$x \notin A\cap B$. 立得$x \in \overline{A \cap B}.$

\end{theo}

(5) 笛卡尔积: $A \times B=\{(x,y)|x \in A, y \in B\}$.

\begin{tcolorbox}[colback=yellow!10,colframe=red!75!black,title=小窍门]
    ~~~~观察德摩根定律的证明, 证两个集合$A$和$B$相等就是证$A$含于$B$且$B$含于$A$. 
    事实上, 证集合相等就这一种办法, 遇到此类题, 这么构思准没错.
  \end{tcolorbox}

  
4. 可列集: 能按某规律排列所有元素的集合. 

\begin{tcolorbox}[colback=gray!5,colframe=orange!75!black,title=注意事项]
    ~~~~可列集合的直观理解是离散无穷, 深入学下去可知道离散无穷"远小于"连续无穷. 常见的整数集和有理数集是可列集, 
    但是实数集不是可列集, 这导致无理数远远多于有理数, 或表述为实数轴几乎处处是无理数.
\end{tcolorbox}

\subsection{映射}

1. 定义: 在两个集合$A$, $B$间定义对应关系$f: A\rightarrow B$, 使得$\forall x \in A$, $\exists y \in B$, $y=f(x)$. 其中
$A$ 称为定义域, $B$称为值域, $f$称为$A$到$B$的映射.

2. 特殊映射: (1) 单射: $\forall x_1,x_2 \in A , f(x_1)\neq f(x_2)$.

(2) 满射: $\forall y \in B, \exists x \in A, y=f(x)$.

(3) 双射: $\forall x \in A ,\exists !y \in B, y=f(x)$. 双射等价于单射+满射.

(4) 逆映射: 令 $f:A\rightarrow B$为双射, 定义 $f^{-1}:B \rightarrow A,$  使得$x=f^{-1}(y)$ 等价于$y=f(x)$.

3. 函数: 数学分析中研究的函数特指实数(欧氏空间)到实数的映射. 定义域指定为$\mathbb{R}$时, $f$称为一元实函数; 定义域指定为$\mathbb{R}^n$时, 
$f$称为多元实(变)函数.

4. 一元函数的表示方式: (1)解析式: 将因变量$y$用自变量$x$表示为 $y=f(x)$.

(2) 参数方程: $\left\{\begin{aligned}
    y&=&\phi(t)\\
    x&=&\psi(t)\\
\end{aligned}\right.$, 其中$t$称为参数.

(3) 隐函数: $F(x,y)=0$, 通常从中解不出显式解析式$y=f(x)$.

5. 函数特性: (1) 单调性. $\forall x_1<x_2, f(x_1)<f(x_2),$ 则$f(x)$单增; $\forall x_1<x_2, f(x_1)>f(x_2),$ 则$f(x)$单减;

(2) 奇偶性. $f(x)=f(-x)$为偶函数, $f(x)=-f(-x)$为奇函数.

(3) 周期性. $\exists T>0, f(x+T)=f(x)$, 则$f(x)$以$T$为周期.

(4) 有界性. $\exists M>0,$ $|f(x)|<M$恒成立, $f(x)$为有界函数.

\begin{tcolorbox}[colback=gray!5,colframe=orange!75!black,title=注意事项]
    ~~~~这部分内容看起来头疼, 一般也不会让你证是不是满射之类的, 很多都无关紧要. 出镜率高的概念有: 逆映射, 参数方程, 隐函数, 单调性和有界性.
\end{tcolorbox}

\section{极限}

\begin{tcolorbox}[colback=red!5,colframe=red!75!black]
~~~~极限是分析理论的基石, 是把微积分概念严谨化不可或缺的概念. 学习极限的最主要目的是熟悉$\varepsilon-N$语言, 并能熟练用这种语言证明.

~~~~什么是极限? 直观上讲就是一列数能够任意逼近某个值. 这个任意性可以用$\varepsilon$概括, 即某项以后, 所有数与该值的距离小于任意正数$\epsilon$. 
用数学语言写出来, 就是极限的$\varepsilon-N$定义.

~~~~怎么用极限概念证明问题? 首先熟悉实数系五大基本定理, 再取定适当的$\varepsilon$和$N$, 使用$\varepsilon-N$语言证明或推翻结论.

\end{tcolorbox}

\subsection{实数系的连续性}

1. 确界: 最小上界称为上确界; 最大下界称为下确界. 数学语言描述: 

$\sup S =\min\{M|\forall x \in S, x \leq M\}$, 
$\inf S =\max\{m|\forall x \in S, x \geq m\}$.

\begin{tcolorbox}[colback=yellow!10,colframe=red!75!black,title=小窍门]
 ~~~~这个定义平时不怎么用, 用得更多的是另一种描述(以上确界为例):

 ~~~~(1) $M$为上界, 即$\forall x \in S, s \leq M$;
 
 ~~~~(2) $M-\varepsilon$都不是上界, 即$\forall \varepsilon>0, \exists x_0 \in S, x_0>M-\varepsilon$.

\end{tcolorbox}

2. 确界存在性定理: 非空有上界的数集必有上确界; 非空有下界的数集必有下确界.

\begin{tcolorbox}[colback=gray!5,colframe=orange!75!black,title=注意事项]
~~~~为什么把确界存在性定理单列? 因为此处把这个定理作为出发点证明其它定理. 证明确界存在性定理的方法是十进制小数逼近, 感觉不那么严谨.
\end{tcolorbox}

\subsection{数列极限}

1. 定义: 对数列$\{a_n\}$, 若$\forall \varepsilon>0, \exists N \in N^*, n>N$时, 恒有$|a_n-A|<\varepsilon$, 
则数列$\{a_n\}$的极限为$A$, 记作$\lim\limits_{n\rightarrow \infty} a_n=A$.