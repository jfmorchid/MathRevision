\chapter{高等代数}
\thispagestyle{empty}

\setlength{\fboxrule}{0pt}\setlength{\fboxsep}{0cm}
\noindent\shadowbox{
\begin{tcolorbox}[arc=0mm,colback=lightblue,colframe=darkblue,title=Advanced Algebra]
\kai{~~~~高等代数同为大一新生必修的数院基础课, 与数分一样, 是多门后继学科最基本的理论基础. 山大的高代分为两个学期讲授: 
第一个学期主要学习线性代数基础, 第二个学期主要学习线性变换和矩阵分析. 高代2被称为挂科神课不无道理, 
相比前一学期内容跨度相当大, 瞬间抽象起来, 引入的线性空间和线性映射等抽象概念很难理解. 但是学得深入又会感觉非常有趣, 
只要满足八大性质的集合, 就能用一组基和常见的欧氏空间和坐标联系起来, 充分体现了数学的抽象之美. }\\
\kai{~~~~学习高等代数, 不需要任何先修知识, 但是要学会将复杂的事物抽象化描述(最好的例子:线性方程组改写为$\boldsymbol{A}\boldsymbol{x}=\boldsymbol{b}$). 
初学阶段, 多思考, 多举例子, 多做题对提高很有帮助.}

\kai{~~~~高代1的重点: 多项式, 行列式, 矩阵, 线性方程组, 二次型}

\kai{~~~~高代2的重点: 线性空间, 线性映射与线性变换, 特征值与特征向量, $\lambda$矩阵与Jordan标准形, 欧氏空间, 正交变换}

\end{tcolorbox}}
\setlength{\fboxrule}{1pt}\setlength{\fboxsep}{4pt}


\newpage

\section{多项式}

\subsection{数域与一元多项式}

1. 设$P$是复数集的一个子集,且$P$对四则运算封闭,则称$P$为一个数域。

2. \textbf{一元多项式:}设$n$为非负整数,$a_1,\cdots,a_n\in P$,则称
\begin{equation*}
    f(x)=a_nx^n+a_{n-1}x^{n-1}+\cdots+a_1x+a_0
\end{equation*}
为数域$P$上的一元多项式。

3. \textbf{多项式的相等:}若$f(x)=a_nx^n+\cdots+a_1x+a_0$,$g(x)=b_nx^n+\cdots+b_1x+b_0$,
且$a_i=b_i$,则称$f(x)$与$g(x)$相等,记作$f(x)=g(x)$。

4. \textbf{多项式的次数:}若$a_n\neq 0$,则称$f(x)$为$n$次多项式,记作$\partial(f)=n$,且称$a_n$为$f(x)$的首项系数。当$a_n=1$时,称$f(x)$为首一多项式。

5. \textbf{一元多项式环:}所有数域$P$上的一元多项式全体,记作$P[x]$。

\subsection{一元多项式的因式}

1. \textbf{多项式除法:}若$\partial(f)\geqslant \partial(g)$,且存在$q(x),r(x)\in P[x]$,使得$f(x)=q(x)\cdot g(x)+r(x)$,满足
$\partial(r)<\partial(g)$,则称$q(x)$为$g(x)$除$f(x)$的商,$r(x)$为$g(x)$除$f(x)$的余。
对固定的$f(x)$和$g(x)$,商和余均是唯一的。

2. \textbf{整除:}若$r(x)=0$,则称$g(x)$整除$f(x)$,记作$g(x)\mid f(x)$。此时称$g(x)$为$f(x)$的因式。

3. \textbf{公因式:}若$\exists h(x)\in P[x]$,使得$h(x)\mid f(x)$且$h(x)\mid g(x)$,则称$h(x)$为$f(x)$和$g(x)$的公因式。

4. \textbf{最大公因式:}若$h(x)$为$f(x)$和$g(x)$的公因式,且对任意$f(x)$和$g(x)$的公因式$h_1(x)$,均有$h_1(x)\mid h(x)$,则称$h(x)$为$f(x)$的最大公因式,记作$h=(f,g)$。

5. \textbf{最大公因式表示定理:}若$h=(f,g)$,则$\exists u(x),v(x)\in P[x]$,使$h(x)=u(x)\cdot f(x)+v(x)\cdot g(x)$。

6. \textbf{Euclid辗转相除法:}设$f(x),g(x)\in P[x]$,以如下步骤求最大公因式:

~~~~(1)若$\partial(f)=\partial(g)$,则通过数乘和减法使得$\partial(f)>\partial(g)$;

~~~~(2)用商和余表示:$f(x)=q(x)\cdot g(x)+r(x)$;

~~~~(3)若$r(x)=0$,则$g(x)$即为最大公因式,退出程序;

~~~~(4)否则,用$g(x)$代替$f(x)$,$r(x)$代替$g(x)$,返回第(2)步。

7. \textbf{互素多项式:}若$(f,g)=1$,则称$f(x)$和$g(x)$为互素多项式。

\subsection{多项式的因式分解}

1. \textbf{不可约多项式:}因式只有$1$和本身的多项式。

2. \textbf{因式分解定理:}$\forall f(x)\in P[x]$,且$\partial(f)>1$,则存在唯一分解式
\begin{equation*}
    f(x)=c\cdot p_1(x)\cdot p_2(x)\cdots p_s(x)
\end{equation*}
其中$p_i(x)$均为首一不可约多项式,$c\in P$为常数。

3. \textbf{重因式:}若$r^k(x)\mid f(x)$,而$r^{k+1}(x)\nmid f(x)$,则称$r(x)$为$f(x)$的$k$重因式。

4. \textbf{代数基本定理:}对每个次数$\geqslant 1$的复系数多项式,在复数域上一定有一个一次因式。

\subsection{有理系数多项式}

1. \textbf{定义:}$\mathbb{Q}[x]$上的多项式。

2. \textbf{性质:}每个次数$\geqslant 1$的有理系数多项式都能唯一分解为不可约有理系数多项式的乘积。

3. \textbf{本原多项式:}在$g(x)=b_nx^n+\cdots+b_1x_1+b_0$中,若$b_n,\cdots,b_0$互素,则称$g(x)$为本原多项式。

4. \textbf{求整系数多项式全部有理根:}若$f(x)\in \mathbb{Z}[x]$,而$\frac{r}{s}$为一个有理根(其中$r$,$s$互素),则必有$r\mid a_0$,$s\mid a_n$,
枚举全部组合再验证即可。

5. \textbf{Eisenstein判别法:}设$f(x)\in \mathbb{Z}[x]$,若存在素数$p$,使得

~~~~(1)$p \nmid a_n$;(2)$p\nmid a_i$,$i=1,2,\cdot,n-1$;(3)$p^2\nmid a_n$,

则$f(x)$在有理数上不可约。

\section{向量与线性方程组}

\subsection{线性方程组}

1. \textbf{定义:}形如
\begin{equation*}
    \left\{\begin{aligned}
        a_{11}x_1&+&\cdots&+&a_{1n}x_n&=&b_1\\
        a_{21}x_1&+&\cdots&+&a_{2n}x_n&=&b_2\\
        &\vdots&&\vdots&&\vdots&\\
        a_{m1}x_1&+&\cdots&+&a_{mn}x_n&=&b_m
    \end{aligned}\right.
\end{equation*}
的方程组称为$n$元线性方程组。

2. \textbf{齐次线性方程组:}当常数项$b_1=\cdots=b_n=0$时,称上述方程组为

3. \textbf{线性方程组的解:}若存在一组实数$(x_1,\cdots,x_n)$,使得全部$m$个等式成立,则称$(x_1,\cdots,x_n)$为线性方程组的一组解。线性方程组全部解的集合称为解集。

4. \textbf{初等变换:}有下列三种形式:

~~~~(1)用某一非零的数乘以该方程;

~~~~(2)把一个方程的倍数加到另一个方程;

~~~~(3)互换两个方程的位置。

初等变换不改变方程组的解集。

5. \textbf{高斯消元法:}将线性方程组化为阶梯型(下式中令$m>n$):
\begin{equation*}
    \left\{\begin{aligned}
        a_{11}x_1&+&a_{12}x_2&+&\cdots&+&a_{1n}x_n&=&b_1\\
        &&a_{22}x_2&+&\cdots&+&a_{2n}x_n&=&b_2\\
        &&&&&&\vdots&&\vdots\\
        &&&&&&a_{nn}x_n&=&b_n\\
        &&&&&&0&=&0\\
        &&&&&&&&\vdots\\
        &&&&&&0&=&0
    \end{aligned}\right.
\end{equation*}
自下而上依次求解$x_n,x_{n-1},\cdots,x_1$。

6. \textbf{线性方程组解的个数:}称“$0=0$”以外的方程为有效方程,个数为$r$。

~~~~(1)无解:若存在方程“$0=c$”,其中常数$c\neq 0$,则方程组无解;

~~~~(2)恰有一组解:若$r=n$,则方程组恰有一组解;

~~~~(3)无穷多组解:若$r<n$,则方程组有无穷多组解,每组解均能用$n-r$个自由变量表示。

\subsection{向量空间}

1. \textbf{向量:}$\mathbb{R}^n$中的元素$\boldsymbol{x}=(x_1,\cdots,x_n)$称为向量,可用于表示$n$元线性方程组的一组解。

2. \textbf{$n$维向量空间:}在$\mathbb{R}^n$上定义加法和数乘构成的空间。令$\boldsymbol{x}=(x_1,\cdots,x_n)$,$\boldsymbol{y}=(y_1,\cdots,y_n)$,则

~~~~(1)加法:$\boldsymbol{x}+\boldsymbol{y}=(x_1+y_1,\cdots,x_n+y_n)$;

~~~~(2)数乘:$\lambda x =(\lambda x_1,\cdots,\lambda x_n)$;

~~~~(3)点积:$\boldsymbol{x}\cdot \boldsymbol{y}=x_1y_1+x_2y_2+\cdots+x_ny_n$。

3. \textbf{线性组合:}若$\exists k_1,\cdots,k_n\in \mathbb{R}$,使$\boldsymbol{\beta} = k_1\boldsymbol{\alpha}_1+\cdots+k_n\boldsymbol{\alpha}_n$,则称$\boldsymbol{\beta}$为$\boldsymbol{\alpha}_1,\cdots,\boldsymbol{\alpha}_n$的一个线性组合,
也称$\boldsymbol{\beta}$可被$\boldsymbol{\alpha}_1,\cdots,\boldsymbol{\alpha}_n$线性表示。

4. \textbf{线性相关:}若存在不全为零的$k_1,\cdots,k_n\in \mathbb{R}$,使得
\begin{equation*}
    k_1\boldsymbol{\alpha}_1+\cdots+k_n\boldsymbol{\alpha}_n=0
\end{equation*}
则称$\boldsymbol{\alpha}_1,\cdots,\boldsymbol{\alpha}_n$线性相关。

5. \textbf{线性无关:}若$k_1\boldsymbol{\alpha}_1+\cdots+k_n\boldsymbol{\alpha}_n=0\Rightarrow k_1=k_2=\cdots=k_n=0$,则称$\boldsymbol{\alpha}_1,\cdots,\boldsymbol{\alpha}_n$线性无关。

6. \textbf{极大线性无关组:}若$\boldsymbol{\alpha}_1,\cdots,\boldsymbol{\alpha}_n$线性无关,且$\forall k = n+1,\cdots,m$,均有$\boldsymbol{\alpha}_1,\cdots,\boldsymbol{\alpha}_n,\boldsymbol{\alpha}_k$线性相关,
则称$\boldsymbol{\alpha}_1,\cdots,\boldsymbol{\alpha}_n$为向量组$\boldsymbol{\alpha}_1,\cdots,\boldsymbol{\alpha}_n,\boldsymbol{\alpha}_{n+1},\cdots,\boldsymbol{\alpha}_m$的一个极大线性无关组,
其中的向量个数称为$\boldsymbol{\alpha}_1,\cdots,\boldsymbol{\alpha}_m$的秩,记作rank$(\boldsymbol{\alpha}_1,\cdots,\boldsymbol{\alpha}_m)=n$。

7. \textbf{重要结论$^*$:}$n$维向量组的秩不超过$n$,即$n$个线性无关的$n$维向量必定可以线性表示所有$n$维向量。

\subsection{线性方程组解的结构}

1. \textbf{解的性质:}同一方程组的解经过加法、数乘运算,仍是原方程组的解。

2. \textbf{基础解系:}一组线性无关的解向量$\boldsymbol{\eta}_1,\cdots,\boldsymbol{\eta}_t$,且能线性表示方程组的所有解。

3. \textbf{导出组:}将$b_1,\cdots,b_n$换为$0$,得到对应的齐次线性方程组称为原方程组的导出组。

4. \textbf{非齐次线性方程组的通解:}设$\boldsymbol{\gamma}_0$为原方程组的一组解,则所有解可表示为$\boldsymbol{\gamma}=\boldsymbol{\gamma}_0+\boldsymbol{\eta}$,其中$\boldsymbol{\eta}$是导出组的任意解。

\section{矩阵与行列式}

\subsection{矩阵的概念}

1. \textbf{定义:}由$m$个$n$维行向量组成的数表称为$m\times n$矩阵,记作
\begin{equation*}
    \boldsymbol{A}=\begin{pmatrix}
        a_{11}&a_{12}&\cdots&a_{1n}\\
        a_{21}&a_{22}&\cdots&a_{2n}\\
        \vdots&\vdots&&\vdots\\
        a_{m1}&a_{m2}&\cdots&a_{mn}
    \end{pmatrix}
\end{equation*}
为了表述方便,通常还会使用如下两种表述矩阵的方式:

~~~~(1)元素、下标表示法:$\boldsymbol{A}=(a_{ij})_{m\times n}$。

~~~~(2)列向量表示法:令$\boldsymbol{A}$的$n$个列向量分别为$\boldsymbol{\alpha}_j=(a_{1j},a_{2j},\cdots,a_{mj})^T$,则
矩阵$\boldsymbol{A}$可表示为$\boldsymbol{A}=(\boldsymbol{\alpha}_1,\cdots,\boldsymbol{\alpha}_n)$。

2. \textbf{矩阵的加法:}令$\boldsymbol{A}=(a_{ij})_{m\times n}$,$\boldsymbol{B}=(b_{ij})_{m\times n}$,$\boldsymbol{C}=\boldsymbol{A}+\boldsymbol{B}$,则$c_{ij}=a_{ij}+b_{ij}$。

3. \textbf{矩阵的数乘:}令$\lambda \in \mathbb{R}$,$\boldsymbol{A}=(a_{ij})_{m\times n}$,$\boldsymbol{B}=\lambda \cdot A$,则$b_{ij}=\lambda \cdot a_{ij}$。

4. \textbf{矩阵乘法:}令$\boldsymbol{A}=(a_{ij})_{m\times n}$,$\boldsymbol{B}=(b_{ij})_{n\times s}$,$C=AB$,则
\begin{equation*}
    C=(c_{ij})_{m\times s}\text{,}c_{ij}=\sum\limits_{k=1}^n a_{ik}\cdot b_{kj}
\end{equation*}
可以简单理解为$\boldsymbol{C}$的第$i$行第$j$列由$\boldsymbol{A}$的第$i$个行向量和$\boldsymbol{B}$的第$j$个列向量点积得到。

5. \textbf{方阵:}$n$行$n$列的矩阵称为$n$阶方阵。如常见的$n$阶单位矩阵:
\begin{equation*}
    E_n=\begin{pmatrix}
        1&0&\cdots&0\\
        0&1&\cdots&0\\
        \vdots&\vdots&&\vdots\\
        0&0&\cdots&1
    \end{pmatrix}
\end{equation*}

6. \textbf{转置:}令$\boldsymbol{A}=(a_{ij})_{m\times n}$,称$\boldsymbol{B}=(b_{ij})_{n\times m}$为$\boldsymbol{A}$的转置,记作$\boldsymbol{B}=\boldsymbol{A}^T$,其中$b_{ij}=a_{ji}$。
若$\boldsymbol{A}^T=\boldsymbol{A}$,则称$\boldsymbol{A}$为对称方阵。

\subsection{行列式}

1. 二阶行列式:
\begin{equation*}
    \left|\begin{matrix}
        a&b\\
        c&d
    \end{matrix}\right|=ad-bc
\end{equation*}
代表向量$(a,b)$和向量$(c,d)$所夹平行四边形的有向面积。

2. \textbf{$n$阶行列式:}称$n$阶方阵$\boldsymbol{A}$的行列式为
\begin{equation*}
    \text{det}\boldsymbol{A}=\left|\begin{matrix}
        a_{11}&a_{12}&\cdots&a_{1n}\\
        a_{21}&a_{22}&\cdots&a_{2n}\\
        \vdots&\vdots&&\vdots\\
        a_{n1}&a_{n2}&\cdots&a_{nn}
    \end{matrix}\right|
\end{equation*}

3. 行列式的性质:

~~~~(1)上三角矩阵的行列式为对角线乘积,即
\begin{equation*}
    \text{det}\boldsymbol{A}=\left|\begin{matrix}
        a_{11}&a_{12}&\cdots&a_{1n}\\
        0&a_{22}&\cdots&a_{2n}\\
        \vdots&\vdots&&\vdots\\
        0&0&\cdots&a_{nn}
    \end{matrix}\right|=\prod\limits_{i=1}^n a_{ii}
\end{equation*}

~~~~(2)将方阵作转置,行列式不变,即det$\boldsymbol{A}=$det$\boldsymbol{A}^T$;

~~~~(3)若$\boldsymbol{A}$的某一行向量为$\boldsymbol{0}$,则det$\boldsymbol{A}=0$;

~~~~(4)互换$\boldsymbol{A}$的两行,det$\boldsymbol{A}$变号;

~~~~(5)将$\boldsymbol{A}$的某一行的$k$倍加到另一行,det$\boldsymbol{A}$不变。

4. \textbf{行列式的求法:}按如下步骤进行。

~~~~(1)若$a_{11}=0$,即$\boldsymbol{A}$的第一列为$\boldsymbol{0}$,由$\boldsymbol{A}^T$有一行为$\boldsymbol{0}$知,det$\boldsymbol{A}=0$;

~~~~(2)否则,若$a_{11}=0$,则存在$a_{j1}\neq 0$,交换第$1$行和第$j$行后,$a_{11}\neq 0$,det$\boldsymbol{A}$变号;

~~~~(3)将第$i$行($i>1$)减去第一行的$\frac{a_{i1}}{a_{11}}$倍,之后$a_{i1}=0$;

~~~~(4)对右下角的$(n-1)\times (n-1)$方阵作相同处理,返回第($1$)步。

经上述处理后,方阵$\boldsymbol{A}$被化为上三角方阵,直接求对角元素乘积即可。

5. \textbf{行列式按行展开:}将det($\boldsymbol{A}$)按第$i$行展开,有
\begin{equation*}
    \text{det} \boldsymbol{A} =\sum\limits_{j=1}^n a_{ij}A_{ij}
\end{equation*}
其中$A_{ij}=(-1)^{i+1}M_{ij}$称为代数余子式,$M_{ij}$为划去第$i$行和第$j$列后剩余部分的行列式。同样可将det$\boldsymbol{A}$按第$j$列展开如下:
\begin{equation*}
    \text{det}\boldsymbol{A}=\sum\limits_{i=1}^n a_{ij}A_{ij}
\end{equation*}

\subsection{在解线性方程组中的应用}

1. \textbf{方程组的矩阵表示:}令系数矩阵、解向量、常数向量分别为
\begin{equation*}
    A=\begin{pmatrix}
        a_{11}&a_{12}&\cdots&a_{1n}\\
        a_{21}&a_{22}&\cdots&a_{2n}\\
        \vdots&\vdots&&\vdots\\
        a_{m1}&a_{m2}&\cdots&a_{mn}\\
    \end{pmatrix}\text{,}\boldsymbol{x}=\begin{pmatrix}
        x_1\\
        x_2\\
        \vdots\\
        x_n
    \end{pmatrix}\text{,}\boldsymbol{b}=\begin{pmatrix}
        b_1\\
        b_2\\
        \vdots\\
        b_n
    \end{pmatrix}
\end{equation*}
则线性方程组可表示为
\begin{equation*}
    \boldsymbol{A}\boldsymbol{x}=\boldsymbol{b}
\end{equation*}

2. \textbf{克莱姆法则:}当$\boldsymbol{A}$为方阵时,令$d_i=\text{det}(\boldsymbol{\alpha}_1,\boldsymbol{\alpha}_2,\cdots,\boldsymbol{\alpha}_{i-1},\boldsymbol{b},\boldsymbol{\alpha}_{i+1},\cdots,\boldsymbol{\alpha}_n)$,$d=\text{det}\boldsymbol{A}$。
若$d\neq 0$,则线性方程组有唯一解$x_i=\frac{d_i}{d}$。

3. \textbf{线性方程组:线性方程组有解判别定理:}方程组$\boldsymbol{A}\boldsymbol{x}=\boldsymbol{b}$有解的充要条件是
\begin{equation*}
    \text{rank}(\boldsymbol{\alpha}_1,\cdots,\boldsymbol{\alpha}_n)=\text{rank}(\boldsymbol{\alpha}_1,\cdots\boldsymbol{\alpha}_n,\boldsymbol{b})
\end{equation*}
即系数矩阵的秩等于增广矩阵的秩。

\subsection{矩阵的性质}

1. \textbf{矩阵的秩:}即列向量组$(\boldsymbol{\alpha}_1,\cdots,\boldsymbol{\alpha}_n)$的秩。同一矩阵行向量组和列向量组的秩相等。

2, 矩阵乘积的性质:当$\boldsymbol{A}$,$\boldsymbol{B}$均为$n\times n$矩阵时,$|\boldsymbol{A}\boldsymbol{B}|=|\boldsymbol{A}|\cdot|\boldsymbol{B}$;rank$(\boldsymbol{A}\boldsymbol{B})\leqslant\text{rank}\boldsymbol{B}$。 

3. \textbf{退化方阵:}当det$(\boldsymbol{A})=0$时,称方阵$\boldsymbol{A}$为退化的。

4. \textbf{矩阵的逆:}若存在方阵$\boldsymbol{B}$,使得$\boldsymbol{A}\boldsymbol{B}=\boldsymbol{B}\boldsymbol{A}=\boldsymbol{E}$,则称$\boldsymbol{B}$为方阵$\boldsymbol{A}$的逆矩阵,记作$\boldsymbol{A}^{-1}$。

5. \textbf{求矩阵的逆:}当$\boldsymbol{A}$为非退化方阵时,有两种方法求矩阵$A$的逆:

~~~~(1)对矩阵$[A|E]$作行变换,得到$[E|A^{-1}]$;

~~~~(2)考虑伴随矩阵
\begin{equation*}
    A^*=\begin{pmatrix}
        A_{11}&A_{21}&\cdots&A_{n1}\\
        A_{12}&A_{22}&\cdots&A_{n1}\\
        \vdots&\vdots&&\vdots\\
        A_{1n}&A_{2n}&\cdots&A_{nn}
    \end{pmatrix}
\end{equation*}
其中$A_{ij}$为$a_{ij}$位置的代数余子式。则有如下恒等式成立:
\begin{equation*}
    A^{-1}=\frac{1}{|A|}A^*
\end{equation*}

6. \textbf{初等矩阵:}由单位矩阵$\boldsymbol{E}$经过一次初等变换得到的矩阵。

7. \textbf{矩阵等价:}若$\boldsymbol{A}$,$\boldsymbol{B}$为$m\times n$矩阵,且存在$m$阶可逆方阵$\boldsymbol{P}$和$n$阶可逆方阵$\boldsymbol{Q}$,使得$\boldsymbol{A}=\boldsymbol{P}\boldsymbol{B}\boldsymbol{Q}$,
则称$\boldsymbol{A}$和$\boldsymbol{B}$等价,此时$\boldsymbol{A}$可由$\boldsymbol{B}$经过初等变换得到。

\section{二次型}

\subsection{二次型的矩阵表示}

1. \textbf{二次型:}一个关于$x_1,\cdots,x_n$的二次齐次多项式
\begin{equation*}
    f(x_1,\cdots,x_n)=a_{11}x_1^2+2a_{12}x_1x_2+\cdots+2a_{n,n-1}x_nx_{n-1}+a_{nn}x_n^2
\end{equation*}

2. \textbf{矩阵表示:}令对称矩阵\begin{equation*}
    A=\begin{pmatrix}
        a_{11}&a_{12}&\cdots&a_{1n}\\
        a_{21}&a_{22}&\cdots&a_{2n}\\
        \vdots&\vdots&&\vdots\\
        a_{m1}&a_{m2}&\cdots&a_{mn}\\
    \end{pmatrix}\text{,}\boldsymbol{x}=\begin{pmatrix}
        x_1\\
        x_2\\
        \vdots\\
        x_n
    \end{pmatrix}
\end{equation*}
则上述二次型可用$\boldsymbol{x}^T\boldsymbol{A}\boldsymbol{x}$表示。

3. \textbf{非退化线性替换:}令$\boldsymbol{P}$为$n$阶可逆方阵,作非退化线性替换$\boldsymbol{x}=\boldsymbol{P}\boldsymbol{y}$,
则\begin{equation*}
    \boldsymbol{x}^T\boldsymbol{A}\boldsymbol{x}=\boldsymbol{y}^T(\boldsymbol{P}^T\boldsymbol{A}\boldsymbol{P})\boldsymbol{y}
\end{equation*}

4. \textbf{矩阵合同:}若存在$n$阶可逆方阵$\boldsymbol{P}$,使得$\boldsymbol{B}=\boldsymbol{P}^T\boldsymbol{A}\boldsymbol{P}$,则称矩阵$\boldsymbol{A}$和$\boldsymbol{B}$合同,此时二次型
$\boldsymbol{x}^T\boldsymbol{A}\boldsymbol{x}$可由非退化线性替换$\boldsymbol{x}=\boldsymbol{P}\boldsymbol{y}$化为二次型$\boldsymbol{y}^T\boldsymbol{B}\boldsymbol{y}$。

\subsection{二次型的化简}

1. \textbf{标准形:}形如$d_1x_1^2+d_2x_2^2+\cdots+d_nx_n^2$的平方和形式。

2. \textbf{化简可行性:}任一对称矩阵均合同于某一对角矩阵,即任意二次型均可通过非退化线性替换转化为标准形。

3. 配方法:

~~~~(1)若存在$t$使得$a_{tt}=0$,则按$x_t$为主元,配方后换元以消去$x_t$的一次项;

~~~~(2)若$\forall t$,$a_{tt}=0$,但$a_{ij}\neq 0$。令$x_i=z_i+z_j$,$x_j=z_i-z_j$,其余变量不变,则$a_{ij}x_ix_j$可转化为含$z_i^2$和$z_j^2$的部分,返回$(1)$。

4. \textbf{规范形:} 形如$z_1^2+\cdots+z_p^2-z_{p+1}^2-\cdots-z_r^2$的二次型。任意实二次型可被非退化线性替换转化为唯一规范形。

5. \textbf{惯性指数:}正平方项的个数$p$称为正惯性指数,负平方项的个数$r-p$称为负惯性指数。

6. \textbf{正定二次型:}$\forall \boldsymbol{x} \in \mathbb{R}^n$,$\boldsymbol{x}^T\boldsymbol{A}\boldsymbol{x}>0$,则称二次型$\boldsymbol{x}^T\boldsymbol{A}\boldsymbol{x}$正定。
$n$元实二次型正定的充要条件是正惯性指数等于$n$。

7. \textbf{半正定:}$\forall \boldsymbol{x} \in \mathbb{R}^n$,$\boldsymbol{x}^T\boldsymbol{A}\boldsymbol{x}\geqslant 0$,则称二次型$\boldsymbol{x}^T\boldsymbol{A}\boldsymbol{x}$半正定。

\section{线性空间}

\subsection{线性空间的定义}

1. \textbf{线性空间:}设$V$为非空集合,$P$为数域,在$V$上定义加法和数乘运算:

~~~~(1)加法:$\boldsymbol{\gamma}=\boldsymbol{\alpha}+\boldsymbol{\beta}$,其中$\boldsymbol{\alpha},\boldsymbol{\beta}\in V$。

~~~~(2)数乘:$\boldsymbol{\delta}=k\boldsymbol{\alpha}$,其中$k\in P$,$\alpha\in V$。

且加法具有结合律、交换律、单位元($0$)和逆元(减法);数乘具有结合律,单位元($1$);加法和数乘满足两项分配律,则称$V$为数域$P$上的线性空间,将$V$中的元素称为向量。

2. \textbf{线性空间举例:}分量属于$P$的全体$n$元数组$P^n$;数域$P$上的一元多项式环$P[x]$。

\subsection{基与坐标}

1. \textbf{可继承$R^n$中向量的概念:}线性表示、线性相关、线性无关、线性组合、极大线性无关组、秩。由于之前已经讨论过这些概念,此处不再赘述。

2. \textbf{维数:}若$V$中最多存在$n$个线性无关的向量,则称$V$为$n$维线性空间。

3. \textbf{基:}$n$维线性空间$V$中,$n$个线性无关的向量$\varepsilon_1,\cdots,\varepsilon_n$称为$V$的一组基。使用$\varepsilon_1,\cdots,\varepsilon_n$可以线性表示$V$中的所有向量。

4. \textbf{坐标:}令$\boldsymbol{\alpha}\in V$,则$\boldsymbol{\alpha}$可唯一地表示为如下形式:
\begin{equation*}
    \boldsymbol{\alpha}=x_1\boldsymbol{\varepsilon}_1+\cdots+x_n\boldsymbol{\varepsilon}_n
\end{equation*}
称$(x_1,\cdots,x_n)$为$\boldsymbol{\alpha}$在基$\boldsymbol{\varepsilon}_1,\cdots,\boldsymbol{\varepsilon}_n$下的坐标。

5. 向量的基-坐标表示法:
\begin{equation*}
    \boldsymbol{\alpha}=(\boldsymbol{\varepsilon}_1,\cdots,\boldsymbol{\varepsilon}_n)\begin{pmatrix}
        x_1\\
        x_2\\
        \vdots\\
        x_n
    \end{pmatrix}
\end{equation*}

\subsection{基变换与坐标变换}

1. \textbf{两组基的转换:}令$\boldsymbol{\varepsilon}_1,\cdots,\boldsymbol{\varepsilon}_n$和$\boldsymbol{\varepsilon}_1',\cdots,\boldsymbol{\varepsilon}_n'$为$V$的两组基,且
\begin{equation*}
    \boldsymbol{\varepsilon}_i' =a_{1i}\boldsymbol{\varepsilon}_1+\cdots+a_{ni}\boldsymbol{\varepsilon}_n=(\boldsymbol{\varepsilon}_1,\cdots,\boldsymbol{\varepsilon}_n)\begin{pmatrix}
        a_{1i}\\
        a_{2i}\\
        \vdots\\
        a_{ni}
    \end{pmatrix}
\end{equation*}
则有如下等式成立,其中$\boldsymbol{A}=(a_{ij})_{n\times n}$称为$\boldsymbol{\varepsilon}_1,\cdots,\boldsymbol{\varepsilon}_n$到$\boldsymbol{\varepsilon}_1',\cdots,\boldsymbol{\varepsilon}_n'$的过渡矩阵。
\begin{equation*}
    (\boldsymbol{\varepsilon}_1',\cdots,\boldsymbol{\varepsilon}_n')=(\boldsymbol{\varepsilon}_1,\cdots,\boldsymbol{\varepsilon}_n)\begin{pmatrix}
        a_{11}&a_{12}&\cdots&a_{1n}\\
        a_{21}&a_{22}&\cdots&a_{2n}\\
        \vdots&\vdots&&\vdots\\
        a_{n1}&a_{n2}&\cdots&a_{nn}
    \end{pmatrix}
\end{equation*}

2. \textbf{同一向量在两组基下的坐标变换:}令
\begin{equation*}
    \boldsymbol{x}=(\boldsymbol{\varepsilon}_1,\cdots,\boldsymbol{\varepsilon}_n)\begin{pmatrix}
        x_1\\
        x_2\\
        \vdots\\
        x_n
    \end{pmatrix}    
\end{equation*}
代入$(\boldsymbol{\varepsilon}_1',\cdots,\boldsymbol{\varepsilon}_n')=(\boldsymbol{\varepsilon}_1,\cdots,\boldsymbol{\varepsilon}_n)\boldsymbol{A}$,得到
\begin{equation*}
    x=(\boldsymbol{\varepsilon}_1',\cdots,\boldsymbol{\varepsilon}_n')A^{-1}\begin{pmatrix}
        x_1\\
        x_2\\
        \vdots\\
        x_n
    \end{pmatrix}
\end{equation*}
因此$\boldsymbol{x}$在基$\boldsymbol{\varepsilon}_1',\cdots,\boldsymbol{\varepsilon}_n'$下的坐标为$\boldsymbol{A}^{-1}(x_1,\cdots,x_n)^T$。

\subsection{线性子空间}

1. \textbf{定义:}若$W\subset V$,$W\neq \Phi$,且$W$对加法和数乘运算封闭,则称$W$为$V$的线性子空间。

2. \textbf{生成子空间:}设$\boldsymbol{\alpha},\cdots,\boldsymbol{\alpha}_m\in V$为一组线性无关的向量,则所有能被$\boldsymbol{\alpha}_1,\cdots,\boldsymbol{\alpha}_m$线性表示的向量构成$V$的一个子空间,称为由$\boldsymbol{\alpha}_1,\cdots,\boldsymbol{\alpha}_m$生成的子空间,记作
$L(\boldsymbol{\alpha}_1,\cdots,\boldsymbol{\alpha}_m)$。

3. \textbf{子空间基的扩充定理$^*$:}若$\boldsymbol{\alpha}_1,\cdots,\boldsymbol{\alpha}_m$为子空间$W$的一组基,则必存在$\boldsymbol{\alpha}_1,\cdots,\boldsymbol{\alpha}_m,\boldsymbol{\alpha}_{m+1},\cdots,\boldsymbol{\alpha}_n$是$V$的一组基。

4. \textbf{子空间的交:}若$V_1\subset V$,$V_2\subset V$均为$V$的子空间,则$V_1\cap V_2$也为$V$的子空间。

5. \textbf{子空间的和:}若$V_1\subset V$,$V_2\subset V$均为$V$的子空间,则$\{\boldsymbol{\alpha}_1+\boldsymbol{\alpha}_2|\boldsymbol{\alpha}_1\in V_1,\boldsymbol{\alpha}_2\in V_2\}$也为$V$的子空间,记作$V_1+V_2$。

6. \textbf{维数公式$^*$:}记dim$V$为$V$的维数,则有如下公式成立:
\begin{equation*}
    \text{dim}V_1+\text{dim}V_2=\text{dim}(V_1+V_2)+\text{dim}(V_1\cap V_2)
\end{equation*}

7. \textbf{子空间的直和:}若$V_1\cap V_2=\{\boldsymbol{0}\}$,则$V_1$和$V_2$的和被称为直和,记作$V_1\oplus V_2$。

8. \textbf{直和的性质$^*$:}$V_1+V_2=V_1\oplus V_2 \Leftrightarrow \text{dim}(V_1+V_2)=\text{dim}V_1+\text{dim}V_2$。

\section{线性变换}

\subsection{映射与变换}

1. \textbf{映射:}某一对应规则$f:A\rightarrow B$,使得$\forall x \in A$,$\exists ! y \in B$,$y=f(x)$。此时称$y$为$x$在映射$f$下的像,$x$称为原像。

2. \textbf{变换:}称集合$A$到自身的映射$f:A\rightarrow A$为$A$到自身的变换。

3. \textbf{线性变换:}保持加法和数乘运算的变换,用$\mathscr{A}$,$\mathscr{B}$等花体字母表示,即
\begin{equation*}
    \mathscr{A}(\boldsymbol{\alpha}+\boldsymbol{\beta})=\mathscr{A}\boldsymbol{\alpha}+\mathscr{A}\boldsymbol{\beta}\text{;}\mathscr{A}(k\boldsymbol{\alpha})=k\cdot \mathscr{A}\boldsymbol{\alpha}
\end{equation*}

\subsection{线性变换的运算}

1. \textbf{线性变换的乘法:}$\mathscr{A}\mathscr{B}(\boldsymbol{\alpha})=\mathscr{A}\left[\mathscr{B}\boldsymbol{\alpha}\right]$。
乘法满足结合律,但一般不满足交换律。

2. \textbf{线性变换的加法:}$(\mathscr{A}+\mathscr{B})\boldsymbol{\alpha}=\mathscr{A}\boldsymbol{\alpha}+\mathscr{B}\boldsymbol{\alpha}$。
加法满足交换律和结合律。

3. \textbf{逆变换:}若$\mathscr{A}\mathscr{B}=\mathscr{B}\mathscr{A}=\mathscr{E}$,其中$\mathscr{E}$为恒等变换,则称$\mathscr{B}$为$\mathscr{A}$的逆变换。

\subsection{线性变换的矩阵表示}

1. \textbf{线性变换与向量的对应关系$^*$:}设$\boldsymbol{\varepsilon},\cdots,\boldsymbol{\varepsilon}_n$为$V$的一组基,则$\forall \boldsymbol{\alpha}_1,\cdots,\boldsymbol{\alpha}_n \in V$,$\exists !\mathscr{A}$,使得$\mathscr{A}\boldsymbol{\varepsilon}_i=\boldsymbol{\alpha}_i$。

2. \textbf{矩阵表示:}将$\mathscr{A}\boldsymbol{\varepsilon}_i$用$\boldsymbol{\varepsilon}_1,\cdots,\boldsymbol{\varepsilon}_n$线性表示:
\begin{equation*}
    \mathscr{A}\boldsymbol{\varepsilon}_i=a_{1i}\boldsymbol{\varepsilon}_1+a_{2i}\boldsymbol{\varepsilon}_2+\cdots+a_{ni}\boldsymbol{\varepsilon}_n=(\boldsymbol{\varepsilon}_1,\cdots,\boldsymbol{\varepsilon}_n)\begin{pmatrix}
        a_{1i}\\
        a_{2i}\\
        \vdots\\
        a_{ni}
    \end{pmatrix}
\end{equation*}
则有\begin{equation*}
    \mathscr{A}(\boldsymbol{\varepsilon}_1,\cdots,\boldsymbol{\varepsilon}_n)=(\mathscr{A}\boldsymbol{\varepsilon}_1,\cdots,\mathscr{A}\boldsymbol{\varepsilon}_n)
    =(\boldsymbol{\varepsilon}_1,\cdots,\boldsymbol{\varepsilon}_n)\begin{pmatrix}
        a_{11}&a_{12}&\cdots&a_{1n}\\
        a_{21}&a_{22}&\cdots&a_{2n}\\
        \vdots&\vdots&&\vdots\\
        a_{n1}&a_{n2}&\cdots&a_{nn}
    \end{pmatrix}
\end{equation*}
将矩阵$\boldsymbol{A}=(a_{ij})_{n\times n}$称为$\mathscr{A}$在基$\boldsymbol{\varepsilon}_1,\cdots,\boldsymbol{\varepsilon}_n$下的矩阵,记作
\begin{equation*}
    \mathscr{A}(\boldsymbol{\varepsilon_1},\cdots,\boldsymbol{\varepsilon}_n)=(\boldsymbol{\varepsilon}_1,\cdots,\boldsymbol{\varepsilon}_n)\boldsymbol{A}
\end{equation*}

3. \textbf{同一线性变换在不同基下的矩阵:}设$\mathscr{A}(\boldsymbol{\varepsilon}_1,\cdots,\boldsymbol{\varepsilon}_n)=(\boldsymbol{\varepsilon}_1,\cdots,\boldsymbol{\varepsilon}_n)\boldsymbol{A}$,同时有
$\mathscr{A}(\boldsymbol{\varepsilon}_1',\cdots,\boldsymbol{\varepsilon}_n')=(\boldsymbol{\varepsilon}_1',\cdots,\boldsymbol{\varepsilon}_n')\boldsymbol{B}$,而$(\boldsymbol{\varepsilon}_1,\cdots,\boldsymbol{\varepsilon}_n)=(\boldsymbol{\varepsilon}_1',\cdots,\boldsymbol{\varepsilon}_n')\boldsymbol{P}$,
则第一个等式两边可分别化为
\begin{eqnarray*}
    \mathscr{A}(\boldsymbol{\varepsilon}_1,\cdots,\boldsymbol{\varepsilon}_n)&=&\mathscr{A}(\boldsymbol{\varepsilon}_1',\cdots,\boldsymbol{\varepsilon}_n')\boldsymbol{P}=(\boldsymbol{\varepsilon}_1',\cdots,\boldsymbol{\varepsilon}_n')\boldsymbol{BP}\\
    (\boldsymbol{\varepsilon}_1,\cdots,\boldsymbol{\varepsilon}_n)\boldsymbol{A}&=&(\boldsymbol{\varepsilon}_1',\cdots,\boldsymbol{\varepsilon}_n')\boldsymbol{PA}
\end{eqnarray*}
得到$\boldsymbol{BP}=\boldsymbol{PA}$,即$\boldsymbol{A}=\boldsymbol{P}^{-1}\boldsymbol{BP}$。

4. \textbf{相似矩阵:}设$\boldsymbol{A}$,$\boldsymbol{B}$均为$n$阶方阵。若存在可逆的$n$阶方阵$\boldsymbol{P}$,使$\boldsymbol{A}=\boldsymbol{P}^{-1}\boldsymbol{BP}$,则称
矩阵$\boldsymbol{A}$与矩阵$\boldsymbol{B}$相似,记作$\boldsymbol{A}\sim\boldsymbol{B}$。相似矩阵的本质是同一线性变换在不同基下的矩阵。

\subsection{特征值与特征向量}

1. \textbf{定义:}令$\mathscr{A}$是$V$上的线性变换,若存在$\lambda \in P$和$\boldsymbol{\xi} \in V$,使得
\begin{equation*}
    \mathscr{A}\boldsymbol{\xi}=\lambda \boldsymbol{\xi}
\end{equation*}
则称$\lambda$为$\mathscr{A}$的一个特征值,$\boldsymbol{\xi}$为$\mathscr{A}$属于特征值$\lambda$的特征向量。

2. \textbf{求解:}设$\mathscr{A}$在某组基下的矩阵为$\boldsymbol{A}$,解$\boldsymbol{A}$的特征多项式的所有根,即关于$\lambda$的方程:
\begin{equation*}
    |\lambda \boldsymbol{E}-\boldsymbol{A}|=0
\end{equation*}
解出所有符合条件的$\lambda$并逐一代入,解线性方程组:
\begin{equation*}
    (\lambda \boldsymbol{E}-\boldsymbol{A})\boldsymbol{x}=0
\end{equation*}
解得的基础解系便是属于$\lambda$的特征向量。

3. \textbf{性质$^*$:}相似的矩阵有相同的特征值,即$\boldsymbol{A}=\boldsymbol{P}^{-1}\boldsymbol{B}\boldsymbol{P}$时,$|\lambda\boldsymbol{E}-\boldsymbol{A}|=|\lambda \boldsymbol{E}-\boldsymbol{B}|$。
换句话说,同一线性变换在不同基下的特征值不变。

4. Hamilton-Cayley定理,记$\boldsymbol{A}$的特征多项式为$f(\lambda)=|\lambda \boldsymbol{E}-\boldsymbol{A}|$,则$f(\boldsymbol{A})=0$。

5. \textbf{对角矩阵的转化:}若$\mathscr{A}$存在$n$个线性无关的特征向量,则$\mathscr{A}$在某组基下的矩阵是对角形的。显然若$\mathscr{A}$的特征多项式有$n$个不同的根,则$\mathscr{A}$在当前基下的矩阵可相似对角化。

6. \textbf{相似对角化方法:}令对角矩阵
\begin{equation*}
    \boldsymbol{\Lambda}=\begin{pmatrix}
        \lambda_1&0&\cdots&0\\
        0&\lambda_2&\cdots&0\\
        \vdots&\vdots&&\vdots\\
        0&0&\cdots&\lambda_n
    \end{pmatrix}
\end{equation*}
对应特征值$\lambda_i$的特征向量为$\boldsymbol{\xi}_i$,则令矩阵$\boldsymbol{P}=(\boldsymbol{\xi}_1,\cdots,\boldsymbol{\xi}_n)$,则有$\boldsymbol{A}\boldsymbol{P}=\boldsymbol{P}\boldsymbol{\Lambda}$,即$\boldsymbol{P}^{-1}\boldsymbol{A}\boldsymbol{P}=\boldsymbol{\Lambda}$。

\subsection{线性变换的值域与核}

1. \textbf{定义:}令$\{\boldsymbol{y}|\exists \boldsymbol{x} \in V, \mathscr{A}\boldsymbol{x}=\boldsymbol{y}\}$为线性变换$\mathscr{A}$的值域,记作$\mathscr{A}V$或Im$\mathscr{A}$;
令$\{\boldsymbol{x}\in V | \mathscr{A}\boldsymbol{x}=\boldsymbol{0}\}$为线性变换$\mathscr{A}$的核,记作$\mathscr{A}^{-1}(\boldsymbol{0})$或ker$\mathscr{A}$。

2. \textbf{秩和零度:}$\mathscr{A}V$的维数称为$\mathscr{A}$的秩;$\mathscr{A}^{-1}(\boldsymbol{0})$的维数称为$\mathscr{A}$的零度。

3. \textbf{重要定理$^*$:}令$\mathscr{A}(\boldsymbol{x}_1),\cdots,\mathscr{A}\boldsymbol{x}_m$为$\mathscr{A}V$的一组基,$\boldsymbol{x}_{m+1},\cdots,\boldsymbol{x}_n$为$\mathscr{A}^{-1}(\boldsymbol{0})$的一组基,则$\boldsymbol{x}_1,\cdots,\boldsymbol{x}_m,\boldsymbol{x}_{m+1},\cdots,\boldsymbol{x}_n$为$V$的一组基,由此可得
\begin{equation*}
    \text{dim}(\mathscr{A}V)+\text{dim}\left(\mathscr{A}^{-1}(\boldsymbol{0})\right)=\text{dim}V
\end{equation*}

\subsection{不变子空间}

1. \textbf{定义:}设$W \subset V$,若$\forall \boldsymbol{\xi}\in W $,均有$\mathscr{A}\boldsymbol{\xi}\in W$,则称$W$为$\mathscr{A}$的不变子空间。

2. \textbf{线性空间的分解:}设$\mathscr{A}$的特征多项式$f(\lambda)$可分解为一次因式的乘积:
\begin{equation*}
    f(\lambda)=\prod\limits_{i=1}^s(\lambda-\lambda_i)^{r_i}
\end{equation*}
则$V$可分解为若干不变子空间的直和$V=V_1\oplus V_2\cdots\oplus V_s$,其中$V_i=\{\boldsymbol{\xi}\in V|(\mathscr{A}-\lambda_i\mathscr{E})^{r_i}\boldsymbol{\xi}=0\}$,该定理用于将矩阵分解为分块准对角形。

\section{λ-矩阵}

\subsection{基本概念}

1. \textbf{定义:}若矩阵$\boldsymbol{A}$中的元素是$\lambda$的多项式,则称$\boldsymbol{A}$为$\lambda$-矩阵,记作$\boldsymbol{A}(\lambda)$。

2. \textbf{可逆:}若$\exists \boldsymbol{B}(\lambda)$,使得$\boldsymbol{A}(\lambda)\boldsymbol{B}(\lambda)=\boldsymbol{B}(\lambda)\boldsymbol{A}(\lambda)=\boldsymbol{E}$,则称$\boldsymbol{B}(\lambda)$是$\boldsymbol{A}(\lambda)$的逆矩阵,记作$\boldsymbol{A}^{-1}(\lambda)$。

3. \textbf{可逆的充要条件:}$\boldsymbol{A}(\lambda)$可逆$\Leftrightarrow |\boldsymbol{A}(\lambda)|$为非零常数。

\subsection{标准形}

1. \textbf{初等变换:}有以下三种形式:

~~~~(1)互换两行/两列的位置;

~~~~(2)某行/列乘以非零常数$c$;

~~~~(3)某行/列加上另一行/列的$\varphi(\lambda)$倍。

2. \textbf{等价:}若$\boldsymbol{A}(\lambda)$可由一系列初等变换得到$\boldsymbol{B}(\lambda)$,则称$\boldsymbol{A}(\lambda)$与$\boldsymbol{B}(\lambda)$等价。

3. \textbf{标准形:}任意$s\times n$的$\lambda$-矩阵$\boldsymbol{A}(\lambda)$均可化为下述形式的矩阵:
\begin{equation*}
    \begin{pmatrix}
        d_1(\lambda)&0&0&\cdots&0&0\\
        0&d_2(\lambda)&0&\cdots&0&0\\
        \vdots&\vdots&\vdots&&\vdots&\vdots\\
        0&0&\cdots&d_r(\lambda)&\cdots&0\\
        0&0&0&\cdots &0&0\\
       \vdots&\vdots&\vdots&&\vdots&\vdots\\
        0&0&0&\cdots&0&0
    \end{pmatrix}
\end{equation*}
其中$r\geqslant 1$,$d_i(\lambda)$为首一多项式,且$d_i(\lambda)\mid d_{i+1}(\lambda)$。该形式称为$\lambda$-矩阵的标准形。

4. 化为标准形的步骤:

~~~~(1)调换行、列,使$a_{11}(\lambda)$是其它所有元素的公因子;

~~~~(2)从$a_{11}$出发,通过初等变换,使得$a_{1j}$和$a_{i1}$均化为$0$;

~~~~(3)对右下角$(n-1)\times (n-1)$的矩阵块重复上述操作。

\subsection{相似不变量}

1. \textbf{不变因子:}标准形中主对角线上非零元素$d_1(\lambda),d_2(\lambda),\cdots,d_r(\lambda)$称为$\boldsymbol{A}(\lambda)$的不变因子。

2. \textbf{初等因子:}将所有不变因子分解为一次首一多项式的乘积,所有一次因式的方幂称为矩阵$\boldsymbol{A}(\lambda)$的初等因子。

3. \textbf{行列式因子:}$\boldsymbol{A}(\lambda)$中全部$k$阶子式的首一最大公因式$D_k(\lambda)$称为$\boldsymbol{A}(\lambda)$的$k$阶行列式因子。有如下等式成立:
\begin{equation*}
    D_k(\lambda)=d_1(\lambda)d_2(\lambda)\cdots d_k(\lambda)
\end{equation*}

\subsection{数字矩阵的相似标准形}

1. \textbf{数字矩阵相似的条件:}$\boldsymbol{A}\sim \boldsymbol{B}\Leftrightarrow \lambda\boldsymbol{E}-\boldsymbol{A}$和$\lambda\boldsymbol{E}-\boldsymbol{B}$等价。

2. \textbf{若尔当标准形:}形为
\begin{equation*}
    \boldsymbol{J}_i(\lambda)=\begin{pmatrix}
        \lambda&0&\cdots&0&0\\
        1&\lambda&\cdots&0&0\\
        0&1&\cdots&0&0\\
        \vdots&\vdots&&\vdots&\vdots\\
        0&0&\cdots&\lambda&0\\
        0&0&\cdots&1&\lambda
    \end{pmatrix}
\end{equation*}
的矩阵称为若尔当块。将由若干若尔当块组成的矩阵\begin{equation*}
    \begin{pmatrix}
        \boldsymbol{J}_1&0&\cdots&0\\
        0&\boldsymbol{J}_2&\cdots&0\\
        \vdots&\vdots&&\vdots\\
        0&0&\vdots&\boldsymbol{J}_n
    \end{pmatrix}
\end{equation*}
称为若尔当形矩阵。

3. \textbf{矩阵化为若尔当标准形:}设$\lambda\boldsymbol{E}-\boldsymbol{A}$的初等因子为$(\lambda-\lambda_1)^{k_1},\cdots,(\lambda-\lambda_s)^{k_s}$,则每个初等因子$(\lambda-\lambda_i)^{k_i}$对应一个若尔当块$\boldsymbol{J_i}(\lambda_i)$,将所有$\boldsymbol{J}_i$组合便得到$\boldsymbol{A}$的若尔当标准形,且除了若尔当块的
排列方式以外,该标准形唯一。

4. \textbf{友矩阵:}设多项式$d(\lambda)=\lambda^n+a_1\lambda^{n-1}+\cdots+a_n$,则称
\begin{equation*}
    \boldsymbol{A}=\begin{pmatrix}
        0&0&\cdots&0&-a_n\\
        1&0&\cdots&0&-a_{n-1}\\
        0&1&\cdots&0&-a_{n-2}\\
        \vdots&\vdots&&\vdots&\vdots\\
        0&0&\cdots&1&-a_1
    \end{pmatrix}
\end{equation*}
为$d(\lambda)$的友矩阵。

5. \textbf{有理标准形矩阵:}称准对角矩阵
\begin{equation*}
    \boldsymbol{A}=\begin{pmatrix}
        \boldsymbol{A}_1&0&\cdots&0\\
        0&\boldsymbol{A}_2&\cdots&0\\
        \vdots&\vdots&&\vdots\\
        0&0&\cdots&\boldsymbol{A}_n
    \end{pmatrix}
\end{equation*}
为有理标准形矩阵,其中$\boldsymbol{A}_i$为$d_i(\lambda)$的友矩阵,且$d_i(\lambda)\mid d_{i+1}(\lambda)$。

6. \textbf{矩阵的有理标准形:}令$\lambda \boldsymbol{E}-\boldsymbol{A}$的不变因子为$1,1,\cdots,d_1(\lambda),\cdots,d_s(\lambda)$,
且$\boldsymbol{A}_i$为$d_i(\lambda)$的友矩阵,便得到矩阵$\boldsymbol{A}$的有理标准形。

\section{欧几里得空间}

\subsection{欧氏空间与内积}

1. \textbf{内积:}设$V$是$\mathbb{R}$上的线性空间,在$V$上定义二元实函数$(\boldsymbol{\alpha},\boldsymbol{\beta})$,满足如下条件:

~~~~(1)正定性:$(\boldsymbol{\alpha},\boldsymbol{\alpha})\geqslant 0$,当且仅当$\boldsymbol{\alpha}=\boldsymbol{0}$时取等;

~~~~(2)对称性;$(\boldsymbol{\alpha},\boldsymbol{\beta})=(\boldsymbol{\beta},\boldsymbol{\alpha})$;

~~~~(3)线性性:$(k_1\boldsymbol{\alpha}+k_2\boldsymbol{\beta},\boldsymbol{\gamma})=k_1(\boldsymbol{\alpha},\boldsymbol{\gamma})+k_2(\boldsymbol{\beta},\boldsymbol{\gamma})$。
则称$(\boldsymbol{\alpha},\boldsymbol{\beta})$为$V$上的内积,定义了内积的线性空间称为欧几里得空间。

2. \textbf{欧氏空间$\mathbb{R}^n$:}在$\mathbb{R}^n$上,令$\boldsymbol{\alpha}=(a_1,\cdots,a_n)$,$\boldsymbol{\beta}=(b_1,\cdots,b_n)$,定义如下内积:
\begin{equation*}
    (\boldsymbol{\alpha},\boldsymbol{\beta})=a_1b_1+\cdots+a_nb_n
\end{equation*}
则$\mathbb{R}^n$构成一最常用的欧氏空间,下文若无特殊说明均在此空间上讨论。

3. \textbf{Cauchy-Schwarz不等式:}$|(\boldsymbol{\alpha,\boldsymbol{\beta}})|\leqslant |\boldsymbol{\alpha}|\cdot |\boldsymbol{\beta}|$,当$\boldsymbol{\alpha}=k\boldsymbol{\beta}$时取等。此不等式对任一符合定义的内积均适用。

4. \textbf{向量夹角:}规定向量$\boldsymbol{\alpha}$和$\boldsymbol{\beta}$的夹角为
\begin{equation*}
    <\boldsymbol{\alpha},\boldsymbol{\beta}>=\arccos \frac{(\boldsymbol{\alpha},\boldsymbol{\beta})}{|\boldsymbol{\alpha}|\cdot |\boldsymbol{\beta}|}
\end{equation*}

5. \textbf{模:}定义$\|\boldsymbol{\alpha}\|=(\boldsymbol{\alpha},\boldsymbol{\alpha})$为向量$\boldsymbol{\alpha}$的模,又称为$\boldsymbol{\alpha}$的长度。

6. \textbf{正交:}若$(\boldsymbol{\alpha},\boldsymbol{\beta})=0$,则称$\boldsymbol{\alpha}$和$\boldsymbol{\beta}$正交,记作$\boldsymbol{\alpha}\perp \boldsymbol{\beta}$。

7. \textbf{度量矩阵:}令$\boldsymbol{\varepsilon}_1,\cdots,\boldsymbol{\varepsilon}_n$为$\mathbb{R}^n$上的一组基,$\boldsymbol{x}=x_1\boldsymbol{\varepsilon}_1+\cdots+x_n\boldsymbol{\varepsilon}_n$,$\boldsymbol{y}=y_1\boldsymbol{\varepsilon}_1+\cdots+y_n\boldsymbol{\varepsilon}_n$,则有
\begin{equation*}
    (\boldsymbol{x},\boldsymbol{y})=\sum\limits_{i=1}^n \sum\limits_{j=1}^n x_iy_j(\boldsymbol{\varepsilon}_i,\boldsymbol{\varepsilon}_j)=\boldsymbol{x}^T\boldsymbol{A}\boldsymbol{y}
\end{equation*}
其中方阵
\begin{equation*}
    \boldsymbol{A}=\begin{pmatrix}
        (\boldsymbol{\varepsilon}_1,\boldsymbol{\varepsilon}_1)&(\boldsymbol{\varepsilon}_1,\boldsymbol{\varepsilon}_2)&\cdots&(\boldsymbol{\varepsilon}_1,\boldsymbol{\varepsilon}_n)\\
        (\boldsymbol{\varepsilon}_2,\boldsymbol{\varepsilon}_1)&(\boldsymbol{\varepsilon}_2,\boldsymbol{\varepsilon}_2)&\cdots&(\boldsymbol{\varepsilon}_2,\boldsymbol{\varepsilon}_n)\\
        \vdots&\vdots&&\vdots\\
        (\boldsymbol{\varepsilon}_n,\boldsymbol{\varepsilon}_1)&(\boldsymbol{\varepsilon}_n,\boldsymbol{\varepsilon}_2)&\cdots&(\boldsymbol{\varepsilon}_n,\boldsymbol{\varepsilon}_n)
    \end{pmatrix}
\end{equation*}
称为度量矩阵。

\subsection{标准正交基}

1. \textbf{正交向量组:}若$\boldsymbol{\alpha}_1,\cdots,\boldsymbol{\alpha}_m$中,$\boldsymbol{\alpha}_i\perp \boldsymbol{\alpha}_j$对$\forall i \neq j$成立,则称$\boldsymbol{\alpha}_1,\cdots,\boldsymbol{\alpha}_m$为正交向量组。注意正交向量组总是线性无关的。

2. \textbf{标准正交基:}若一组基$\boldsymbol{\varepsilon}_1,\cdots,\boldsymbol{\varepsilon}_n$两两正交,且$|\boldsymbol{\varepsilon}_i|=1$恒成立,则称$\boldsymbol{\varepsilon}_1,\cdots,\boldsymbol{\varepsilon}_n$为标准正交基。

3. \textbf{标准正交基的性质:}度量矩阵为单位矩阵,即$(\boldsymbol{x},\boldsymbol{y})=\boldsymbol{x}^T\boldsymbol{y}$。

4. \textbf{施密特正交化:}已知$\boldsymbol{\alpha}_1,\cdots,\boldsymbol{\alpha}_n$为$\mathbb{R}^n$上一组基,取
\begin{eqnarray*}
    \boldsymbol{\beta}_1=\boldsymbol{\alpha}_1,\boldsymbol{\beta}_2=\boldsymbol{\alpha}_2-\frac{(\boldsymbol{\alpha_2},\boldsymbol{\beta_1})}{(\boldsymbol{\beta}_1,\boldsymbol{\beta}_1)}\boldsymbol{\beta}_1,\cdots,\\
    \boldsymbol{\beta}_n=\boldsymbol{\alpha}_n-\frac{(\boldsymbol{\alpha}_n,\boldsymbol{\beta}_1)}{(\boldsymbol{\beta}_1,\boldsymbol{\beta}_1)}\boldsymbol{\beta}_1-\cdots-\frac{(\boldsymbol{\alpha}_n,\boldsymbol{\beta}_{n-1})}{(\boldsymbol{\beta}_{n-1},\boldsymbol{\beta}_{n-1})}\boldsymbol{\beta}_{n-1}
\end{eqnarray*}
则$\frac{\boldsymbol{\beta}_1}{|\boldsymbol{\beta}_1|},\cdots,\frac{\boldsymbol{\beta}_n}{|\boldsymbol{\beta}_n|}$为一组标准正交基。

\subsection{正交变换}

1. \textbf{定义:}保持向量内积不变的变换:
\begin{equation*}
    (\mathscr{A}\boldsymbol{\alpha},\mathscr{A}\boldsymbol{\beta})=(\boldsymbol{\alpha},\boldsymbol{\beta})
\end{equation*}

2. \textbf{性质$^*$:}正交变换保持向量长度不变;在任一组标准正交基下的矩阵是正交矩阵。

3. \textbf{正交矩阵:}满足$\boldsymbol{A}\boldsymbol{A}^T=\boldsymbol{E}$的矩阵。注意标准正交基构成的矩阵$(\boldsymbol{\varepsilon}_1,\cdots,\boldsymbol{\varepsilon}_n)$为正交矩阵。

4. \textbf{正交子空间:}若$\forall \boldsymbol{\alpha} \in V_1,\boldsymbol{\beta}\in V_2$,均有$(\boldsymbol{\alpha},\boldsymbol{\beta})=0$,则称$V_1$和$V_2$正交。

5. \textbf{正交补:}若$V_1\perp V_2$,且$V_1+V_2=V$,则称$V_1$为$V_2$的正交补。

\subsection{实对称矩阵的正交对角化}

1. \textbf{目标:}对实对称矩阵$\boldsymbol{A}$,寻找正交矩阵$\boldsymbol{T}$,使得$\boldsymbol{T}^{-1}\boldsymbol{A}\boldsymbol{T}$为对角矩阵。

2. 步骤:

~~~~(1)求出实对称矩阵$\boldsymbol{A}$的特征值;

~~~~(2)对每个特征值,求解对应特征向量;

~~~~(3)将特征向量施密特正交化并单位化,排列为正交矩阵$\boldsymbol{T}$,此时$\boldsymbol{T}^{-1}\boldsymbol{A}\boldsymbol{T}$正好为对角矩阵,且对角元素是矩阵$A$的所有特征值,按特征向量的顺序排列。