\chapter{解析几何}
\thispagestyle{empty}

\setlength{\fboxrule}{0pt}\setlength{\fboxsep}{0cm}
\noindent\shadowbox{
\begin{tcolorbox}[arc=0mm,colback=lightblue,colframe=darkblue,title=Mathematical Analysis]
\kai{~~~~数学分析是大一新生所修的重要学科基础课, 相比非数学专业更强调证明, 对收敛性的讨论篇幅较大, 与大二的实变函数课程联系紧密. 数分
是今后多门专业课的先修课程: 积分学应用于概率论对随机变量的研究; 对积分的进一步研究(Lebesgue积分)是实变函数的重要内容; Fourier变换和多元函数积分学是偏微分方程必不可少的工具... 
山大主选教材为陈纪修的《数学分析》(第三版), 在此基础上结合卓里奇的数学分析教程, 对共计三个学期的数分课程进行完整的内容回顾.}\\

\kai{~~~~数分1的重点: 极限与连续概念, 一元函数微分学, 微分中值定理}\\

\kai{~~~~数分2的重点: 一元函数积分学, 数项级数和函数项级数, 广义积分}\\

\kai{~~~~数分3的重点: 多元函数微分学, 含参变量积分, 多元函数积分学(重积分, 曲线与曲面积分)}

\end{tcolorbox}}
\setlength{\fboxrule}{1pt}\setlength{\fboxsep}{4pt}


\newpage

\section{极限与连续概念}