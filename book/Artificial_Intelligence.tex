\chapter{人工智能}
\thispagestyle{empty}

\setlength{\fboxrule}{0pt}\setlength{\fboxsep}{0cm}
\noindent\shadowbox{
\begin{tcolorbox}[arc=0mm,colback=lightblue,colframe=darkblue,title=Artificial Intelligence]
\kai{~~~~人工智能是一门综合型强、应用范围广的课程,近年来大火的机器学习/深度学习、大数据技术、智能家居,都是人工智能这一庞大课题下的分支。简要来说,人工智能就是用算法实现人类的部分“智能”,比较常见的有思考、表达、学习、行为等能力,
由此延伸出了计算机视觉、机器学习、数据挖掘等算法,稳定性和效率甚至大幅超越了人类的处理水平。
}\\

\kai{~~~~学习人工智能,需要先修离散数学、概率论,并熟练掌握一门程序语言,用代码实现来印证理论。
重点内容为:数理逻辑基础、确定性与不确定性推理、启发式搜索、随机搜索、机器学习与深度学习。 }\\

\kai{~~~~课程展示内容仅为人工智能领域的冰山一角,近年来延伸出了诸如计算机视觉等活跃研究领域,可通过相关专著进行知识扩展。
}\\

\end{tcolorbox}}
\setlength{\fboxrule}{1pt}\setlength{\fboxsep}{4pt}


\newpage

\section{基本概念}

\begin{tcolorbox}[colback=red!5,colframe=red!75!black]
\kai{~~~~这一部分主要阐述了人工智能“是什么”,通过研究人工智能的目的,引申出对应的技术目标和研究领域,为阐述具体的人工智能算法作铺垫。}
\end{tcolorbox}

\subsection{人工智能的目标}

1. 含义:用计算机算法实现人类的部分智能。

2. 感知能力:模拟人的视觉和听觉,对输入的音频、图像进行分析。与计算机视觉和计算机听觉对应。

3. 思考能力:运用逻辑思维、形象思维,对数据进行推理、论证。与机器证明和自然语言处理对应。

4. 学习能力:从先验信息中“学习”数据的规律,并用于推导后验信息。与机器学习和知识表示对应。

5. 行为能力:能通过一定的方式对外界产生反馈。与智能控制和智能机器人相对应。

6. 图灵测试:衡量人工智能水平的准则。若仅通过对答无法分辨AI端是否为真人,则称该AI端通过了图灵测试。

\subsection{人工智能的部分内容}

1. 知识表示:用特定的形式描述人类知识。书中介绍了谓词逻辑、产生式和框架式三种基本的表示方法。

2. 确定性推理:使用数理逻辑,从确定的数据出发,推导出确定的结论。为了证明某子句是不可满足的,书中额外提出了鲁宾孙归结原理。

3. 不确定性推理:使用概率论,从不精确的数据出发,推导出不确定的结论。书中介绍了可信度方法、证据理论和模糊推理方法。

4. 搜索策略:按照某种顺序对状态空间进行遍历,以局部最优近似表示整体最优的策略。书中除了介绍常规的广搜、深搜,还介绍了启发式搜索策略。

5. 计算智能:受到自然规律启发,通过大量模拟和计算,产生局部最优解。书中介绍了进化算法、遗传算法、粒子群算法等。

6. 专家系统:在特定的领域内,运用专家积累的知识和经验,解决专家试图求解的问题。书中主要介绍了专家系统的工作原理。

\section{数理逻辑基础}

\begin{tcolorbox}[colback=red!5,colframe=red!75!black]
    \kai{~~~~人工智能的推理涉及大量的数理逻辑知识。鉴于数院未开设专门的离散数学课,此处对基本的数理逻辑进行补充,以降低后续章节的理解难度。参考教材:Rosen离散数学第七版中文版。}
\end{tcolorbox}

\subsection{命题逻辑}

1. 命题:一个陈述语句,只能为真(T)或假(F),即拥有确定的真值。

2. 否定形式:命题$p$与其否定形式的真值相反,记作$\neg p$。

3. 联结词:通过多个已知命题构建新命题的逻辑运算符。

~~~~(1)合取:$p$且$q$,当且仅当$p,q$均为真时“$p$且$q$”为真,记作$p \wedge q$。

~~~~(2)析取:$p$或$q$,只需$p,q$有一个为真时“$p$或$q$”为真,记作$p \vee q$。

~~~~(3)异或:当且仅当$p,q$一真一假时,“$p$异或$q$”为真,记作$p \oplus q$。

4. 条件语句: $p$蕴含$q$,当且仅当$p$为真$q$为假时,“$p$蕴含$q$”为假,记作$p \rightarrow q$。注意$p \rightarrow q$等价于命题“$q \vee \neg p$”。

5. 条件语句的延伸:

~~~~(1)逆命题:“$p \rightarrow q$”的逆命题为$q \rightarrow p$。

~~~~(2)否命题:“$p \rightarrow q$”的否命题为$\neg p \rightarrow \neg q$。

~~~~(3)逆否命题:“$p \rightarrow q$”的逆否命题为$\neg q \rightarrow \neg p$。任意命题和其逆否命题的真值均相同。

~~~~(4)双向蕴含:$p$当且仅当$q$,记作$p \leftrightarrow q$。仅$p,q$真值一致时,$p \leftrightarrow q$为真。

6. 真值表:列出命题变元的所有可能情形,并计算对应复合命题的真值,列成如下所示的表格:

\begin{table}[H]
    \centering
    \resizebox{.5\textwidth}{!}{%
    \begin{tabular}{|c|c|c|c|c|c|}
    \hline
    $p$ & $q$ & $p\rightarrow q$ & 复合命题2 & ... & 复合命题$n$ \\ \hline
    T & T & T & ... & ... & ... \\ \hline
    T & F & F & ... & ... & ... \\ \hline
    F & T & T & ... & ... & ... \\ \hline
    F & F & T & ... & ... & ... \\ \hline
    \end{tabular}}
\end{table}

7. 优先级:非 > 或 = 且 > 蕴含

8. De Morgan律:$\neg (p \vee q) = \neg p \wedge \neg q$;$\neg (p \wedge q) = \neg p \vee \neg q$。

\subsection{谓词逻辑}

1. 谓词:表示主语具有的某些性质。如“我是大学生”这个命题中,“是大学生”为谓词。

2. 谓词的数学描述:$n$元谓词用$P(x_1,\cdots,x_n)$表示,本身无真值,但给定变元$x_1,\cdots,x_n$的值后,$P(x_1,\cdots,x_n)$的真值随之确定。

3. 量词:表示何种程度上谓词对一定范围内的个体成立。

~~~~(1)全称量词:对$x$的所有取值,谓词$P(x)$均为真。记作$\forall x P(x)$。

~~~~(2)存在量词:存在$x$的一个取值,使得$P(x)$为真。记作$\exists x P(x)$。

~~~~(3)唯一性量词:有且只有一个$x$,使$P(x)$为真。记作记作$\exists! x P(x)$。
    
~~~~(4)优先级:量词优先于所有逻辑运算符。

~~~~(5)量词De Morgan律:$\neg \forall x P(x)\Leftrightarrow \exists x \neg P(x)$;$\neg \exists x P(x)\Leftrightarrow \forall x \neg P(x)$。

\subsection{推理规则}

1. 论证:从前提推导出结论的过程。

2. 推理规则:利用永真式,得出有效的论证形式。

~~~~(1)假言推理:$\because p$,$ p \rightarrow q$ ,$ \therefore q$。

~~~~(2)取拒式:$\because \neg q$,$ p \rightarrow q$ ,$ \therefore \neg p$。

~~~~(3)假言三段论:$\because p \rightarrow q$,$q \rightarrow r$,$\therefore p \rightarrow r$。

~~~~(4)析取三段论:$\because p \vee q $,$\neg p$,$\therefore q$。

~~~~(5)消解律:$\because p \vee q$,$\neg p \vee r$,$\therefore q \vee r$。

3. 量化命题的推理规则:

~~~~(1)全称实例:$\because \forall x P(x)$,$ \therefore P(c)$。

~~~~(2)全称引入:$\because P(c)$(任意$c$),$ \therefore \forall x P(x)$。

~~~~(3)存在实例:$\because \exists x P(x)$,$ \therefore P(c)$(对某个$c$)。

~~~~(4)存在引入:$\because P(c)$(对某个$c$),$ \therefore \exists x P(x)$。


\section{知识表示}

\begin{tcolorbox}[colback=red!5,colframe=red!75!black]
\kai{~~~~从出生起,人们就在不断接受外部的信息,并通过学习转化为自己掌握的知识。为了拥有人的智能,计算机也需要将外部信息转化为“知识”。
如何储存并利用计算机学习到的知识,就是知识表示的任务。此处主要基于数理逻辑,将知识储存为一系列因果关系即规则。}
\end{tcolorbox}

\subsection{知识表示的任务}

1. 知识:将有关信息关联在一起产生的信息结构,反映事务间的客观关系。

2. 规则:通过因果关系形成的知识。相对地,定义形式的知识称为“事实”。

3. 知识表示:将人类的知识形式化,产生计算机能理解的数据结构。

\subsection{谓词逻辑知识表示}

1. 目标:将知识用谓词逻辑$P(x_1,\cdots,x_n)$表示,其中$x_1,\cdots,x_n$为与知识相关的对象,$P(x_1,\cdots,x_n)$的量词和真值通过知识本身确定。

2. 步骤:首先确定知识的主体,之后提取谓词,最后根据知识的语义描述,确定谓词公式。

\subsection{产生式知识表示}

1. 规则的产生式表示:将规则用因果关系表述,即"IF $p$ THEN $q$",或写作"$p\rightarrow q$"。当规则具有不确定性时,额外注明规则的置信度。

2. 事实的产生式表示:将事实用三元组(对象,属性,值)表述。当事实具有不确定性时,额外注明事实的置信度。

3. 产生式系统:一组产生式形成的系统,可用某一产生式的结论作为另一产生式的前提使用。通常分为规则库、推理机、综合数据库三部分。

~~~~(1)规则库:产生式的集合。

~~~~(2)综合数据库:存放推理中间过程产生的信息。

~~~~(3)推理机:将综合数据库中的中间结论与规则库中的产生式匹配,并激活符合前提条件的产生式。

\subsection{框架表示法}

1. 框架:由若干个“槽”组成,还可以把“槽”分为若干个“侧面”。

2. 槽和侧面:槽用于描述对象的某一属性,侧面用于描述槽属性的一个方面。


