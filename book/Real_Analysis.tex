\chapter{实变函数}
\thispagestyle{empty}

\setlength{\fboxrule}{0pt}\setlength{\fboxsep}{0cm}
\noindent\shadowbox{
\begin{tcolorbox}[arc=0mm,colback=lightblue,colframe=darkblue,title=Mathematical Analysis]
\kai{~~~~实变函数是数学分析的直接后继课,主要学习测度论和Lebesgue积分。相比于数学分析,实变函数的最大特点是抽象、深入且广泛。实变函数研究的是数学分析认为性质“较差”的函数,
由此提出了“几乎处处”的概念;对数学分析中黎曼不可积的函数,提出Lebesgue积分。}\\

\kai{~~~~实变函数的基本理念是将集合论的观点与方法渗入数学分析,基本的论证手法是对特定的集合按某种要求作分解与组合。}\\


\end{tcolorbox}}
\setlength{\fboxrule}{1pt}\setlength{\fboxsep}{4pt}


\newpage

\section{集合论基础}

\subsection{集合}

1. 定义:具有特定性质的事物(即元素)组成的整体。

2. 包含关系:$\forall x \in A$,$x \in B\Rightarrow A \subset B$,记作$A$包含于$B$。

3. 相等关系:$A=B \Leftrightarrow A \subset B$,$B \subset A$。

4. 集合交:$x \in \bigcap\limits_{i=1}^n A_i \equiv \forall i \leq n$,$x \in A_i$。

5. 集合并:$x \in \bigcup\limits_{i=1}^n A_i \equiv \exists i \leq n$,$x \in A_i$。

6. 分配律:$A \cup (B\cap C)=(A\cup B)\cap (A\cup C)$;$A \cap (B\cup C)=(A\cap B)\cup (A\cap C)$。

7. 德摩根律:$(A\cap B)^C =A^C \cup B^C$;$(A\cup B)^C =A^C \cap B^C$。

\subsection{可列个集合的运算}

1. 可列交运算:$x \in \bigcap\limits_{n=1}^\infty A_n \equiv \forall n \in \mathbb{N}^*$,$x \in A_i$。

2. 可列并运算:$x \in \bigcup\limits_{n=1}^\infty A_n \equiv \exists n \in \mathbb{N}^* $,$x \in A_i$。

3. 域(代数):满足下列条件:

~~~~(1)$\Phi \in \mathscr{F}$;(2)$A \in \mathscr{F}\Rightarrow A^C \in \mathscr{F}$;

~~~~(3)$A \in \mathscr{F}$,$B \in \mathscr{F}\Rightarrow A \cup B \in  \mathscr{F}$。

4. $\sigma$-域:$\mathscr{F}$是域的前提下,$A_1,\cdots,A_n\in \mathscr{F}\Rightarrow \bigcup\limits_{n=1}^\infty A_n\in \mathscr{F}$。

5. 上极限:$x \in \limsup\limits_n A_n \equiv$存在无穷多个$n$,$x \in A_n$。

6. 下极限:$x \in \liminf\limits_n A_n \equiv$仅存在有限个$n$,$x \notin A_n$。

7. 上、下极限的数学表述$^*$:

~~~~(1)上极限: $\limsup\limits_n A_n =\bigcap\limits_{m=1}^\infty \bigcup\limits_{i=m}^\infty A_i$;

~~~~(2)下极限: $\liminf\limits_n A_n =\bigcup\limits_{m=1}^\infty \bigcap\limits_{i=m}^\infty A_i$。

8. 收敛集合列:若$\liminf\limits_n A_n=\limsup\limits_n A_n$,则称集合列$\{A_n\}$收敛,并将该极限记为$\lim\limits_n A_n$。


\subsection{集合的基数}

1. 定义:对有限集而言,基数指集合中的元素个数;对无限集而言,基数是关于映射的偏序关系,两者均记作$\overline{A}$。

2. 对等:若$A$和$B$间存在双射$\varphi:A\rightarrow B$,则称$A$和$B$对等,又称$A$和$B$具有相同的基数,记作$A\sim B$。

3. 无穷集合的定义:能与本身一个真子集对等的集合。

4. Cantor定理$^*$:$\mathbb{N}^*$和$[0,1]$不对等。

5. 基数偏序关系:若$A$和$B$不对等,但$A$和$B$的某个真子集对等,则称$A$的基数小于$B$的基数,记作$\overline{A}<\overline{B}$。

6. Bernstein定理$^*$:若$\exists A^* \subset A$,$B^* \subset B$,使得$A\sim B^*$,$B \sim A^*$,则$A\sim B$。

\subsection{可列集合}

1. 定义:能与$\mathbb{N}^*$对等的集合称为可列集合。

2. 充要条件:集合可列的充要条件是元素可被排成一个无穷序列。

3. 常见可列集合$^*$:$\mathbb{Z}$,$\mathbb{Q}$。

4. 重要定理$^*$: 若$A_1,\cdots,A_n$至多可列,则$\bigcup\limits_{n=1}^\infty A_n$至多可列。

5. 不可列集合:不是可列集合的无穷集合。

6. 连续势:与$\mathbb{R}$对等的集合具有连续势。

\section{n维空间中的点集}

\subsection{n维空间与距离}

1. $n$维空间:$n$个实数组成的有序数组$(x_1,\cdots,x_n)$的全体组成的集合,记作$\mathbb{R}^n$。$\mathbb{R}^n$中的元素称为$\mathbb{R}^n$中的点,用粗体字母表示,如$\boldsymbol{x}$。

2. 点的距离:满足下列条件的二元函数$\rho(\boldsymbol{x},\boldsymbol{y})$。

~~~~(1)正定性:$\rho(\boldsymbol{x},\boldsymbol{y})\geqslant 0$,当且仅当$\boldsymbol{x}=\boldsymbol{y}$时,$\rho(\boldsymbol{x},\boldsymbol{y})=0$。

~~~~(2)对称性:$\rho(\boldsymbol{x},\boldsymbol{y})=\rho(\boldsymbol{y},\boldsymbol{x})$。

~~~~(3)三角不等式:$\rho(\boldsymbol{x},\boldsymbol{y})\leqslant \rho(\boldsymbol{x},\boldsymbol{z})+\rho(\boldsymbol{z},\boldsymbol{y})$。

3. 欧几里得距离:最常用的距离,定义为
\begin{equation*}
    \rho(\boldsymbol{x},\boldsymbol{y})=\sqrt{\sum\limits_{i=1}^n(x_i-y_i)^2}。    
\end{equation*}

4. $\delta$邻域:称集合$\{x|\rho(\boldsymbol{x},\boldsymbol{x}_0)<\delta\}$为$\boldsymbol{x}_0$点的$\delta$邻域,记作$O(\boldsymbol{x}_0,\delta)$。

5. 模:记$\boldsymbol{x}$的模为
\begin{equation*}
    \|\boldsymbol{x}\|=\rho(\boldsymbol{x},\boldsymbol{O})=\sqrt{\sum\limits_{i=1}^n x_i^2}
\end{equation*}

\subsection{点与集合的关系}

1. 内点:若$\exists O(\boldsymbol{x}_0,\delta)\subset E$,则$\boldsymbol{x}_0$是集合$E$的内点。

2. 边界点:若$\forall O(\boldsymbol{x}_0,\delta) =E_0$,$\exists \boldsymbol{x}_1 \in E_0$,$\boldsymbol{x}_1 \in E$,且$\exists \boldsymbol{x}_2 \in E_0$,$\boldsymbol{x}_2 \in E$,
则$\boldsymbol{x}_0$是$E$的边界点。简单来说,边界点的任意一个邻域内都有$E$和$E$外的点。

3. 外点:若$\exists O(\boldsymbol{x}_0)\in E^C$,则$\boldsymbol{x}_0$是$E$的外点。

4. 聚点:若$\forall O(\boldsymbol{x}_0,\delta)$,存在无穷个$\boldsymbol{x}\in O(\boldsymbol{x}_0,\delta)$,使得$\boldsymbol{x}\in E$,则$\boldsymbol{x}_0$是$E$的聚点。

5. 孤立点:是$E$的边界点,但不是$E$的聚点。

6. 导集:$E$的所有聚点构成的集合,记作$E'$。

7. 闭包:称$E\cup E'$为$E$的闭包,记作$\overline{E}$。

8. 离散集合:$E'=\Phi$的集合。

9. Bolzano-Weierstrass定理:若$E$是$\mathbb{R}^n$中有界的无穷集合,则$E'\neq \Phi$。

\subsection{集合的分类}

1. 开集:所有点均为其内点的集合。

2. 闭集:满足$E' \subset E$的集合。

3. 重要定理$^*$:

~~~~(1)任意一族闭集之交为闭集,任意一族开集之并未开集。

~~~~(2)有限多个闭集之并为闭集,有限多个开集之交为开集。

4. 博雷尔有限覆盖定理:若$F$为有界闭集,$\mathscr{M}$是一族开邻域,且$\mathscr{M}$完全覆盖$F$,则必存在有限多个
$N_1,\cdots,N_m\in \mathscr{M}$,它们完全覆盖$F$。

5. 完备集:没有孤立点的闭集。

6. Borel集类:在$\mathbb{R}^n$上所有开集的基础上,通过取余集、可列交、可列并构成的全体集合。Borel集类的元素称为Borel集。






